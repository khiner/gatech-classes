\documentclass{article}
%=================================================
% Basics
%=================================================
\usepackage{fixltx2e} % Makes \( \) equation style robust, among other
                      % things. Must be the first package.


% Makes ligatured fonts searchable and copyable in pdf readers
\usepackage{cmap} % Load before fontenc 

% Always include these font encodings in your document 
% unless you have a very good reason.
\usepackage[T1]{fontenc}
\usepackage[utf8]{inputenc}

\usepackage{verbatim}

%=============
% Fonts
%=============

\usepackage{lmodern} % Improved version of computer modern
\usepackage[scale=0.88]{tgheros} % Helvetica clone for sans serif font


\newcommand\hmmax{2} % Default is 3.
\newcommand\bmmax{2} % Default is 4.

\usepackage{bm} % boldmath must be called after the package
\providecommand{\mathbold}[1]{\bm{#1}}

%=============
% AMS Packages and fonts
%=============
\usepackage{amsmath,amsbsy,amsgen,amscd,amsthm,amsfonts,amssymb} 

%=============
% Margins and paper size
%=============
\usepackage[centering,top=1.5in,bottom=1.2in,left=1.4in,right=1.4in]{geometry}
\usepackage{parskip}


%=============
% Section headings
%=============
\usepackage[sf,bf,compact]{titlesec}

%=============
% Tables and lists
%=============
\usepackage{booktabs,longtable,tabu} % Nice tables
\setlength{\tabulinesep}{1mm}
\usepackage[font=small,margin=10pt,labelfont={sf,bf},labelsep={space}]{caption}

%=============
% Code output
%=============
% \usepackage{listings}
% \usepackage{minted}




\usepackage{enumitem}
\setitemize{itemsep=0pt} 
\setenumerate{itemsep=0pt}
\setlist{labelindent=\parindent,%  % Recommended by enumitem package
  font=\sffamily}


%=============
% Hyperlink colors
%=============
\usepackage[usenames,dvipsnames]{xcolor}
\definecolor{steelblue}{HTML}{A1BDC7}
\definecolor{orange}{HTML}{D98C21}
\definecolor{silver}{HTML}{B0ABA8}
\definecolor{rust}{HTML}{B8420F}
\definecolor{seagreen}{HTML}{2E6B69}
\definecolor{joshua}{HTML}{FBDC7F}
\definecolor{darksky}{HTML}{154c79}

\colorlet{steelblue}{silver!30!white}
\colorlet{darkorange}{orange!85!black}
\colorlet{darksilver}{silver!85!black}
\colorlet{darksteelblue}{steelblue!85!black}
\colorlet{darkrust}{rust!85!black}
\colorlet{darkseagreen}{seagreen!85!black}

\usepackage{url}
\usepackage[colorlinks=true]{hyperref}
\hypersetup{linkcolor=darkrust}    
\hypersetup{citecolor=darkseagreen}      
\hypersetup{urlcolor=darksilver}     

%=============
% Microtype
%=============
\usepackage[final]{microtype} 

%=====================
% Header
%=====================
% \usepackage{fancyhdr}
% \usepackage{nopageno} % Gets rid of page number at the bottom
% \fancyhf{} % Clear header style
% \renewcommand{\headrulewidth}{0.5pt} % remove the header rule
% \pagestyle{fancy}
% \fancyhead[LE,RO]{\textsf{\small \thepage}}
% 
% \setlength{\headheight}{14pt}
%=====================
% Fix delimiters
%=====================

% Fixes \left and \right spacing issues. See discussion at
% http://tex.stackexchange.com/questions/2607/spacing-around-left-and-right
\let\originalleft\left
\let\originalright\right
\renewcommand{\left}{\mathopen{}\mathclose\bgroup\originalleft}
\renewcommand{\right}{\aftergroup\egroup\originalright}

%=================================================
% Math macros
%=================================================

%=============
% Generalities
%=============
\usepackage{mathtools}
\mathtoolsset{centercolon}  % Makes := typeset correctly for definitions

%%% Equation numbering
%\numberwithin{equation}{section} 

%%% Annotations
\newcommand{\notate}[1]{\textcolor{red}{\textbf{[#1]}}}

%==============
% Symbols
%==============
\let\oldphi\phi
\let\oldeps\epsilon

\renewcommand{\phi}{\varphi}
\renewcommand{\epsilon}{\varepsilon}
\newcommand{\eps}{\varepsilon}

%==============
% Constants
%==============

% Set constants upright
\newcommand{\cnst}[1]{\mathrm{#1}}  
\newcommand{\econst}{\mathrm{e}}

\newcommand{\zerovct}{\vct{0}} % Zero vector
\newcommand{\Id}{\mathbf{I}} % Identity matrix
\newcommand{\onemtx}{\bm{1}}
\newcommand{\zeromtx}{\bm{0}}

%==============
% Sets
%==============
\providecommand{\mathbbm}{\mathbb} % In case we don't load bbm

% Reals, complex, naturals
\newcommand{\R}{\mathbbm{R}}
\newcommand{\C}{\mathbbm{C}}
\newcommand{\K}{\mathbbm{K}}
\newcommand{\N}{\mathbbm{N}}

%==============
% Probability
%==============
\newcommand{\Prob}{\operatorname{\mathbbm{P}}}
\newcommand{\Expect}{\operatorname{\mathbb{E}}}

%==============
% Vectors and matrices 
%==============
\newcommand{\vct}[1]{\mathbold{#1}}
\newcommand{\mtx}[1]{\mathbold{#1}}

\newcommand{\mrange}{\operatorname{range}}
\newcommand{\mnull}{\operatorname{null}}


\usepackage{graphicx}

\title{HW 2}
\author{Karl Hiner}
\date{\today}

\begin{document}
\maketitle	

Let the input domain be $\mathcal{X}$.
Consider a Hilbert space $\mathbb{H}$, a feature map $\Phi : \mathcal{X} \to \mathbb{H}$ and
a kernel $k(x, x') = \langle \Phi(x), \Phi(x')\rangle$, where $\langle \cdot, \cdot \rangle$ is the inner product on $\mathbb{H}$.

\section{Minimum enclosing ball (MEB) problem}

Consider the following optimization problem for finding the minimum enclosing ball
(MEB) of a set of points $S = \{x_1, \ldots, x_m\} \subset \mathcal{X}$:
$$\min_{r>0, c \in \mathbb{H}}{r^2} \text{ subject to } \|c - \Phi(x_i)\|^2 \leq r^2, i = 1, \dots, m.$$
Show how to derive the dual optimization problem:
$$\max_{\alpha \in \mathbb{R}^m}{\sum_{i=1}^m{\alpha_i k(x_i, x_i)} - \sum_{i=1}^m{\sum_{j=1}^m{\alpha_i\alpha_j k(x_i, x_j)}}} \text{ subject to } \alpha_i \geq 0 \text{ and } \sum_{i=1}^{m}{\alpha_i} = 1, i = 1, \dots, m.$$
Prove that the optimal solution $c = \sum_{i=1}^m{\alpha_i \Phi(x_i)}$ is a convex combination of the features at the training points $x_1, \dots, x_m$.

Hints:
\begin{enumerate}
\item Make the problem finite dimensional in $c$ using the Kernel trick.
Justify this step as done in class.
\item Write down the KKT conditions for the primal problem.
\end{enumerate}

\textbf{Answer:}

\subsection{Make the problem finite dimensional in \textit{c}}

Using the representer theorem, we can express the center \(c\) of the MEB as a linear combination of the mapped data points:
\[ c = \sum_{i=1}^{m} a_i \Phi(x_i) \]
Substituting this expression into the optimization problem gives:
\[ \min_{r>0, a \in \mathbb{R}^m}{r^2} \text{ subject to } \left\| \sum_{j=1}^{m} a_j \Phi(x_j) - \Phi(x_i) \right\|^2 \leq r^2, \quad i = 1, \dots, m \]
\subsection{KKT conditions for the primal problem}

The Lagrangian for this problem is:
\[ L(r, a, \alpha) = r^2 + \sum_{i=1}^{m} \alpha_i \left( \left\| \sum_{j=1}^{m} a_j \Phi(x_j) - \Phi(x_i) \right\|^2 - r^2 \right) \]
Where \( \alpha_i \) are the Lagrange multipliers for the constraints.
Now, we can derive the KKT conditions:
\begin{align*}
\frac{\partial L}{\partial r} &= 0\\
2r - 2r\sum_{i=1}^{m} \alpha_i &= 0\\
\sum_{i=1}^{m} \alpha_i &= 1 \quad \text{(Disregarding the trivial case of $r = 0$)}\\
\frac{\partial L}{\partial a} &= 0 \\
\alpha_i \left( \left\| \sum_{j=1}^{m} a_j \Phi(x_j) - \Phi(x_i) \right\|^2 - r^2 \right) &= 0,\quad \alpha_i \geq 0 \quad i = 1, \dots, m \quad\text{(Complementarity)} \\
\end{align*}
Using the kernel trick and the inner product definition, we can rewrite the squared norm constraint:
\begin{align*}
    \left\| \sum_{j=1}^{m} a_j \Phi(x_j) - \Phi(x_i) \right\|^2 \leq r \\
    \langle \sum_{j=1}^{m} a_j \Phi(x_j) - \Phi(x_i), \sum_{k=1}^{m} a_k \Phi(x_k) - \Phi(x_i) \rangle & & \text{Inner product} \\
    \sum_{j, k=1}^{m} a_j a_k \langle \Phi(x_j), \Phi(x_k) \rangle - 2 \sum_{j=1}^{m} a_j \langle \Phi(x_j), \Phi(x_i) \rangle + \langle \Phi(x_i), \Phi(x_i) \rangle & & \text{Expand} \\
    \sum_{j, k=1}^{m} a_j a_k k(x_j, x_k) - 2 \sum_{j=1}^{m} a_j k(x_j, x_i) + k(x_i, x_i) &  & k(x, y) \coloneqq \langle \Phi(x), \Phi(y) \rangle \\
    k(x_i, x_i) + \sum_{j,k=1}^{m} a_j a_l k(x_j, x_k) - 2 \sum_{j=1}^ {m} a_j k(x_i, x_j) & &\text{Rearrange}
\end{align*}
Substituting this result into the Lagrangian and taking the derivative with respect to $a$:
\begin{align*}
L(r, a, \alpha) &= r^2 + \sum_{i=1}^{m} \alpha_i \left( k(x_i, x_i) + \sum_{j,k=1}^{m} a_j a_k k(x_j, x_k) - 2 \sum_{j=1}^{m} a_j k(x_i, x_j) - r^2 \right) \\
\frac{\partial L}{\partial a_p}  &= 0\\
0 &= \sum_{i=1}^{m} \alpha_i \left( \partial a_p\left(\sum_{j,k=1}^{m} a_j a_k k(x_j, x_k) - 2 \sum_{j=1}^{m} a_j k(x_i, x_j)\right)\right)\quad\text{(Drop terms w/o $a_p$)}\\
&= \sum_{i=1}^{m}\alpha_i \left( 2\sum_{j=1}^{m} a_j k(x_p, x_j) - 2 k(x_i, x_p) \right) \quad \forall p \in [m] \quad \text{(Differentiate)}\\
 &= \sum_{i=1}^{m}\alpha_i \left( \sum_{j=1}^{m} a_j k(x_p, x_j) - k(x_i, x_p) \right) \quad \forall p \in [m] \quad \text{(Simplify)} \\
 &= \sum_{j=1}^{m} a_j k(x_p, x_j) - \sum_{i=1}^{m}\alpha_i k(x_i, x_p), \quad \forall p \in [m] \quad \left(\sum_{i=1}^{m}\alpha_i = 1\right) \\
 &= \sum_{j=1}^{m} a_j k(x_p, x_j) - \sum_{j=1}^{m}\alpha_j k(x_j, x_p), \quad \forall p \in [m] \quad \text{(Change index label)} \\
0 &= \sum_{j=1}^{m} (a_j - \alpha_j)k(x_i, x_j), \quad \forall i \in [m] \quad \text{(Change $p$ index to $i$. $k(\cdot,\cdot)$ is symmetric.)}
\end{align*}
This implies that $a_j = \alpha_j$ for all $j \in [m]$ (ignoring the trivial case where $\forall (i, j), k(x_i,x_j) = 0$). 

Now, let's simplify the Lagrangian, $L(r, a, \alpha)$:
\begin{align*}
r^2 + \sum_{i=1}^{m} \alpha_i \left( k(x_i, x_i) + \sum_{j,k=1}^{m} a_j a_k k(x_j, x_k) - 2 \sum_{j=1}^{m} a_j k(x_i, x_j) - r^2 \right) &\quad\text{(Start)} \\
\sum_{i=1}^{m} \alpha_i \left( k(x_i, x_i) + \sum_{j,k=1}^{m} a_j a_k k(x_j, x_k) - 2 \sum_{j=1}^{m} a_j k(x_i, x_j) \right) + r^2 - \sum_{i=1}^{m} \alpha_i r^2 &\quad\text{(Rearrange)}\\
\sum_{i=1}^{m} \alpha_i \left( k(x_i, x_i) + \sum_{j,k=1}^{m} a_j a_k k(x_j, x_k) - 2 \sum_{j=1}^{m} a_j k(x_i, x_j) \right) &\quad \left(\sum_{i=1}^{m} \alpha_i = 1\right)\\
\sum_{i=1}^{m} \alpha_i k(x_i, x_i) +\sum_{i=1}^{m} \alpha_i  \sum_{j,k=1}^{m} a_j a_k k(x_j, x_k) - 2\sum_{i=1}^{m} \alpha_i  \sum_{j=1}^{m} a_j k(x_i, x_j) &\quad \text{(Distribute)}\\
\sum_{i=1}^{m} \alpha_i k(x_i, x_i) + \sum_{i,j=1}^{m} a_i a_j k(x_k, x_j) - 2\sum_{i=1}^{m} \alpha_i  \sum_{j=1}^{m} a_j k(x_i, x_j) &\quad \left(\sum_{i=1}^{m} \alpha_i = 1\right)\\
\sum_{i=1}^{m} \alpha_i k(x_i, x_i) - \sum_{i=1}^{m}\sum_{j=1}^{m} \alpha_i\alpha_j k(x_i, x_j) &\quad \left(a_j = \alpha_j \forall j \in [m]\right)\\
\end{align*}
To derive the dual problem, we maximize \( L \) with respect to \( \alpha \) (which are our dual variables), and apply the complementarity constraints:
\[ \max_{\alpha \in \mathbb{R}^m} \left\{ \sum_{i=1}^m \alpha_i k(x_i, x_i) - \sum_{i=1}^m \sum_{j=1}^m \alpha_i \alpha_j k(x_i, x_j) \right\} \]
subject to:
\[ \alpha_i \geq 0 \quad \text{and} \quad \sum_{i=1}^{m} \alpha_i = 1, \quad i = 1, \dots, m \]
This is what we wanted to show.

\textbf{Convex combination proof}:
The constraints in the dual problem ensure that \( \alpha_i \geq 0 \) for all \( i \) and that their sum is 1. This implies that each \( \alpha_i \) is a weight in the convex combination. Therefore, the optimal solution \( c = \sum_{i=1}^{m} \alpha_i \Phi(x_i) \) is indeed a convex combination of the features at the training points \( x_1, \dots, x_m \).

\section{Anomaly detection hypothesis class}
Consider the hypothesis class
$$\mathcal{H} = \lbrace h_{c,r}(x) = r^2 - \|c - \Phi(x)\|^2 : \|c\| \leq \Lambda, 0 < r \leq R \rbrace,$$
where $\|\cdot\|$ is the norm induced by the inner product on $\mathbb{H}$, i.e., $\|c\| = \sqrt{\langle c, c \rangle}$.
A hypothesis $h_{c,r}$ is an anomaly detector that flags an input $x$ as an anomaly if $h_{c,r}(x) < 0$.
Show that if $\sup_x{\|\Phi(x)\| < M}$, then the solution to the MEB problem in Part 1 is in $\mathcal{H}$ with $\Lambda \leq M$ and $R \leq 2M$.

Hint: Use the complementarity conditions in Part 1 to get an expression for an optimal $r$ in terms of $\alpha$ and $\Phi(x_i)$.
Now that you have expressions for optimal $c$ and $r$, prove that their norms are upper bounded by $M$ and $2M$ respectively.

\textbf{Answer:}

In Part 1, we showed that we can express $c$ as a linear combination of the mapped data points:
\begin{align*}
    c &= \sum_{i=1}^{m} \alpha_i \Phi(x_i) \\
    \|c\| &= \left\| \sum_{i=1}^{m} \alpha_i \Phi(x_i) \right\| \\
    &\leq \sum_{i=1}^{m} |\alpha_i| \left\| \Phi(x_i) \right\| \quad \text{By the triangle inequality} \\
    &= \sum_{i=1}^{m} \alpha_i \left\| \Phi(x_i) \right\| \quad \text{Since $\alpha_i \geq 0, \forall i \in [m]$} \\
    &\leq \sup_x{\|\Phi(x)\|} \quad \text{Using def. of supremum, and $\sum_{i=1}^{m} \alpha_i = 1$} \\
    &\leq M\quad \text{by assumption}
\end{align*}
Thus, if $c$ in an optimal hypothesis, $\|c\| \leq M$.
Since any $c$ in the hypothesis class $\mathcal{H}$ hs $\|c\| \leq \Lambda$, then any $c$ in the solution to Part 1 is in $\mathcal{H}$ with $\Lambda \leq M$.

Now, we find an expression for the optimal $r$.
From the complementarity conditions in Part 1, we have:
$$\alpha_i \left( \left\| \sum_{j=1}^{m} \alpha_j \Phi(x_j) - \Phi(x_i) \right\|^2 - r^2 \right) = 0.$$
Ignoring the trivial case of \( \alpha_i = 0 \), this gives:
\begin{align*}
    r^2 &= \left\| \sum_{j=1}^{m} \alpha_j \Phi(x_j) - \Phi(x_i) \right\|^2. \\
    &\leq \left( \left\| \sum_{j=1}^{m} \alpha_j \Phi(x_j) \right\| + \|\Phi(x_{i})\| \right)^2 &\text{Triangle inequality}
\end{align*}
Given the constraint \( \sup_x{\|\Phi(x)\| < M} \) and since the \( \alpha \)'s sum to 1 and are non-negative, both terms in the sum can be bounded by \( M \).
Thus,
$$\left( \left\| \sum_{j=1}^{m} \alpha_j \Phi(x_j) \right\| + \|\Phi(x_{i})\| \right)^2 \leq (M + M)^2 = 4M^2.$$
And so, $r^2 \leq 4M^2 \implies r \leq 2M.$
This, since any $r$ in the hypothesis class $\mathcal{H}$ hs $0 < r \leq R$, then any $r$ in the solution to Part 1 is in $\mathcal{H}$ with $R \leq 2M$.

\section{The kernel SVM interpretation}
Let $k(x, x) = 1$, a constant independent of $x$ (this is, e.g., true for the Gaussian kernel).
Derive the following margin-maximization and minimization of the slack penalty $\sum_i\xi$ for finding a hyperplane for this 1-class classification problem:
\begin{equation}\label{eq:4}
\min_{w,\xi}{\frac{1}{2}\|w\|^2 + C\|\xi\|_1} \text{ subject to } \langle w, \Phi(x_i)\rangle \geq 1 - \xi_i, \xi \geq 0, i \in [m].
\end{equation}
Here, all the training points have true labels 1. Suppose $\nu$ is an upper bound on the fraction of support vectors out of m training points.
Equivalently, a maximum of $\nu m$ points are allowed to have $\alpha_i \neq 0$: they could be misclassified as anomalies $(\xi > 1)$ or classified with a nonzero penalty $(1 > \xi_i \geq 0)$ as non-anomalies.

Show that when $C = 1 / (\nu m)$, the above problem is equivalent to MEB in Part 1.
This means that one can equivalently find a hyperplane instead of a minimal enclosing hypersphere in feature space.

Hints:
\begin{enumerate}
    \item Follow the derivation done in class of maximum (geometric) margin classification leading to the soft SVM problem;
    now there is only one label class and the domain space is the feature space, i.e., $x_i \to \Phi(x_i)$.
    \item Show that the dual form of MEB in Part 1 reduces, when $k(x,x) = 1$, to
    $$\min_\alpha\sum_{i,j}^{m}{\alpha_i}{\alpha_j}k(x_i,x_j) \text{ subject to } \alpha_i \geq 0, i \in [m] \text{ and } \sum_{i=1}^{m}{\alpha_i} = 1.$$
    \item Next, derive the dual form of (\ref*{eq:4}) by first writing down the KKT conditions.
    Your results should be very similar to the soft-SVM KKT conditions (5.26-5.30 in Mohri et al).
    \item Now, $\alpha_i = 0$ or $0 < \alpha \leq C$.
    Thus, $\sum_{i=1}^{m}{\alpha_i} \leq C \times$ the number of support vectors.
    Using this, prove that the two dual forms are equivalent.
\end{enumerate}

\textbf{Answer:}

\subsection{Soft SVM Derivation}
We start with the maximum margin classification problem for one-class SVM.
Here, we maximize the geometric margin while penalizing slack variables.
\begin{equation}\label{eq:1}
\min_{w, \xi} \frac{1}{2} \|w\|^2 + C\|\xi\|_1
\end{equation}
subject to:
\begin{align*}
\langle w, \Phi(x_i)\rangle &\geq 1 - \xi_i, \\
\xi &\geq 0,
\end{align*}
for $i \in [m]$, where $\xi_i$ are the slack variables, and $C$ is a constant.

\subsection{Dual form of Minimum Enclosing Ball}
Let's first express the dual form of MEB when $k(x, x) = 1$.
The dual form is
$$\min_\alpha \sum_{i, j=1}^{m} \alpha_i \alpha_j k(x_i, x_j)$$
subject to:
$$\alpha_i \geq 0, \quad i \in [m] \quad \text{and} \quad \sum_{i=1}^{m} \alpha_i = 1.$$

\subsection{KKT Conditions for Equation \ref{eq:1}}
The Lagrangian of the problem in Equation \ref{eq:1} can be formulated as:
$$L(w, \xi, \alpha, \beta) = \frac{1}{2}\|w\|^2 + C\|\xi\|_1 - \sum_{i=1}^{m} \alpha_i (\langle w, \Phi(x_i) \rangle - 1 + \xi_i) - \sum_{i=1}^{m} \beta_i \xi_i,$$
with $\alpha_i \geq 0, \beta_i \geq 0$ and $\xi_i \geq 0$ for $i \in [m]$.

Stationarity conditions:
\begin{align*}
\frac{\partial L}{\partial w} &= w - \sum_{i=1}^{m} \alpha_i \Phi(x_i) = 0 \\
\implies w &= \sum_{i=1}^{m} \alpha_i \Phi(x_i) \\
\frac{\partial L}{\partial \xi_i} &= C - \alpha_i - \beta_i = 0
\end{align*}
Complementarity conditions:
\begin{align*}
\alpha_i (\langle w, \Phi(x_i) \rangle - 1 + \xi_i) &= 0, \\
\beta_i \xi_i &= 0.
\end{align*}
Primal feasibility conditions:
$$\langle w, \Phi(x_i) \rangle \geq 1 - \xi_i, \xi_i \geq 0 \text{ for } i \in [m]$$
Dual feasibility conditions:
$$\alpha_i \geq 0, \beta_i \geq 0 \text{ for } i \in [m]$$

\subsection{Equivalence of Dual forms}
Using the following:
\begin{enumerate}
\item The constraint for the dual form of MEB, $\sum_{i=1}^{m} \alpha_i = 1,$
\item The stationarity condition for \( \xi_i \), \( C - \alpha_i - \beta_i = 0 \), and
\item The dual feasibility condition $\alpha_i \geq 0 \text{ for } i \in [m]$,
\end{enumerate}
we can derive that \( \alpha_i \in [0, C] \).

Due to the upper bound \( \nu \) on the fraction of support vectors, at most \( \nu m \) points can have \( \alpha_i \neq 0 \).
Therefore,
\begin{align*}
\sum_{i=1}^{m} \alpha_i &\leq C \nu m\\
&= 1 \quad \text{(substitute $C = \frac{1}{\nu m}$)},
\end{align*}
which is a relaxation of the MEB constraint \( \sum_{i=1}^{m} \alpha_i = 1 \).

Thus, under \( C = \frac{1}{\nu m} \), the dual form of Equation \ref{eq:1} is a relaxed version of the dual form of MEB.
\end{document}

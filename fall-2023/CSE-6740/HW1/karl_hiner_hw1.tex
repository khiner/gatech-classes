\documentclass{article}
%=================================================
% Basics
%=================================================
\usepackage{fixltx2e} % Makes \( \) equation style robust, among other
                      % things. Must be the first package.


% Makes ligatured fonts searchable and copyable in pdf readers
\usepackage{cmap} % Load before fontenc 

% Always include these font encodings in your document 
% unless you have a very good reason.
\usepackage[T1]{fontenc}
\usepackage[utf8]{inputenc}

\usepackage{verbatim}

%=============
% Fonts
%=============

\usepackage{lmodern} % Improved version of computer modern
\usepackage[scale=0.88]{tgheros} % Helvetica clone for sans serif font


\newcommand\hmmax{2} % Default is 3.
\newcommand\bmmax{2} % Default is 4.

\usepackage{bm} % boldmath must be called after the package
\providecommand{\mathbold}[1]{\bm{#1}}

%=============
% AMS Packages and fonts
%=============
\usepackage{amsmath,amsbsy,amsgen,amscd,amsthm,amsfonts,amssymb} 

%=============
% Margins and paper size
%=============
\usepackage[centering,top=1.5in,bottom=1.2in,left=1.4in,right=1.4in]{geometry}
\usepackage{parskip}


%=============
% Section headings
%=============
\usepackage[sf,bf,compact]{titlesec}

%=============
% Tables and lists
%=============
\usepackage{booktabs,longtable,tabu} % Nice tables
\setlength{\tabulinesep}{1mm}
\usepackage[font=small,margin=10pt,labelfont={sf,bf},labelsep={space}]{caption}

%=============
% Code output
%=============
% \usepackage{listings}
% \usepackage{minted}




\usepackage{enumitem}
\setitemize{itemsep=0pt} 
\setenumerate{itemsep=0pt}
\setlist{labelindent=\parindent,%  % Recommended by enumitem package
  font=\sffamily}


%=============
% Hyperlink colors
%=============
\usepackage[usenames,dvipsnames]{xcolor}
\definecolor{steelblue}{HTML}{A1BDC7}
\definecolor{orange}{HTML}{D98C21}
\definecolor{silver}{HTML}{B0ABA8}
\definecolor{rust}{HTML}{B8420F}
\definecolor{seagreen}{HTML}{2E6B69}
\definecolor{joshua}{HTML}{FBDC7F}
\definecolor{darksky}{HTML}{154c79}

\colorlet{steelblue}{silver!30!white}
\colorlet{darkorange}{orange!85!black}
\colorlet{darksilver}{silver!85!black}
\colorlet{darksteelblue}{steelblue!85!black}
\colorlet{darkrust}{rust!85!black}
\colorlet{darkseagreen}{seagreen!85!black}

\usepackage{url}
\usepackage[colorlinks=true]{hyperref}
\hypersetup{linkcolor=darkrust}    
\hypersetup{citecolor=darkseagreen}      
\hypersetup{urlcolor=darksilver}     

%=============
% Microtype
%=============
\usepackage[final]{microtype} 

%=====================
% Header
%=====================
% \usepackage{fancyhdr}
% \usepackage{nopageno} % Gets rid of page number at the bottom
% \fancyhf{} % Clear header style
% \renewcommand{\headrulewidth}{0.5pt} % remove the header rule
% \pagestyle{fancy}
% \fancyhead[LE,RO]{\textsf{\small \thepage}}
% 
% \setlength{\headheight}{14pt}
%=====================
% Fix delimiters
%=====================

% Fixes \left and \right spacing issues. See discussion at
% http://tex.stackexchange.com/questions/2607/spacing-around-left-and-right
\let\originalleft\left
\let\originalright\right
\renewcommand{\left}{\mathopen{}\mathclose\bgroup\originalleft}
\renewcommand{\right}{\aftergroup\egroup\originalright}

%=================================================
% Math macros
%=================================================

%=============
% Generalities
%=============
\usepackage{mathtools}
\mathtoolsset{centercolon}  % Makes := typeset correctly for definitions

%%% Equation numbering
%\numberwithin{equation}{section} 

%%% Annotations
\newcommand{\notate}[1]{\textcolor{red}{\textbf{[#1]}}}

%==============
% Symbols
%==============
\let\oldphi\phi
\let\oldeps\epsilon

\renewcommand{\phi}{\varphi}
\renewcommand{\epsilon}{\varepsilon}
\newcommand{\eps}{\varepsilon}

%==============
% Constants
%==============

% Set constants upright
\newcommand{\cnst}[1]{\mathrm{#1}}  
\newcommand{\econst}{\mathrm{e}}

\newcommand{\zerovct}{\vct{0}} % Zero vector
\newcommand{\Id}{\mathbf{I}} % Identity matrix
\newcommand{\onemtx}{\bm{1}}
\newcommand{\zeromtx}{\bm{0}}

%==============
% Sets
%==============
\providecommand{\mathbbm}{\mathbb} % In case we don't load bbm

% Reals, complex, naturals
\newcommand{\R}{\mathbbm{R}}
\newcommand{\C}{\mathbbm{C}}
\newcommand{\K}{\mathbbm{K}}
\newcommand{\N}{\mathbbm{N}}

%==============
% Probability
%==============
\newcommand{\Prob}{\operatorname{\mathbbm{P}}}
\newcommand{\Expect}{\operatorname{\mathbb{E}}}

%==============
% Vectors and matrices 
%==============
\newcommand{\vct}[1]{\mathbold{#1}}
\newcommand{\mtx}[1]{\mathbold{#1}}

\newcommand{\mrange}{\operatorname{range}}
\newcommand{\mnull}{\operatorname{null}}



\title{HW 1}
\author{Karl Hiner}
\date{\today}

\begin{document}
\maketitle	

\section{Linear Regression}

\subsection{a}

\begin{equation} \label{eq:1}
    \hat{w} = (X^TX)^{-1}X^TY
\end{equation}
\begin{equation} \label{eq:2}
    y_i = w^T x_i + \epsilon_i,
\end{equation}
where $\epsilon_i \sim \mathcal{N}(0, \sigma^2)$, $w \in \R^d$, and $\{X^i, Y^i\}$ is the $i$-th data point, with $1 \leq i \leq m$.

Using the normal equation (Eqn. \ref{eq:1}), and the model (Eqn. \ref{eq:2}), derive the expectation $\mathbb{E}\left[\hat{w}\right]$.
Note that here $X$ is fixed, and only $Y$ is random.
\begin{align*}
    \mathbb{E}\left[\hat{w}\right] &= \mathbb{E}\left[(X^TX)^{-1}X^TY\right]&\text{Eqn. \ref{eq:1}}\\
    &= \mathbb{E}\left[(X^TX)^{-1}X^T\left(Xw + \epsilon\right)\right]&\text{Substitute $Y$}\\
    &= \mathbb{E}\left[(X^TX)^{-1}X^TXw + (X^TX)^{-1}X^T\epsilon\right]&\text{Distribute}\\
    &= \mathbb{E}\left[w + (X^TX)^{-1}X^T\epsilon\right]&\text{Simplify}\\
    &= w + (X^TX)^{-1}X^T\mathbb{E}\left[\epsilon\right]&\text{Linearity of expectation}\\
    &= w&\text{Since $\mathbb{E}\left[\epsilon\right] = 0$}
\end{align*}

\subsection{b}

Similarly, derive the variance $\text{Var}\left[\hat{w}\right]$.
\begin{align*}
    \text{Var}\left[\hat{w}\right] &= \text{Var}\left[(X^TX)^{-1}X^T\left(Xw + \epsilon\right)\right]&\text{Eqn. \ref{eq:1}, Substitute $Y$}\\
    &= \text{Var}\left[(X^TX)^{-1}X^TXw + (X^TX)^{-1}X^T\epsilon\right]&\text{Distribute}\\
    &= \text{Var}\left[w + (X^TX)^{-1}X^T\epsilon\right]&\text{Simplify}\\
    &= \text{Var}\left[(X^TX)^{-1}X^T\epsilon\right]&\text{Since $w$ is constant}\\
    &= (X^TX)^{-1}X^T\text{Var}\left[\epsilon\right]X(X^TX)^{-1}&\text{$\text{Var}[\vct{b}^T\mtx{X}] = \vct{b}^T\text{Var}[\mtx{X}]\vct{b}$}\\
    &= (X^TX)^{-1}X^T(\sigma^2I_m)X(X^TX)^{-1}&\text{$\text{Var}\left[\epsilon\right] \triangleq \sigma^2$}\\
    &= \sigma^2(X^TX)^{-1}X^TX(X^TX)^{-1}&\text{Commute $\sigma^2$}\\
    &= \sigma^2(X^TX)^{-1}&\text{Simplify}\\
\end{align*}

\subsection{c}

Under the white noise assumption above, does $\hat{w}$ follow a Gaussian distribution with mean and variance in (a) and (b), respectively?
Why or why not?

\textbf{Answer:} Yes, $\hat{w}$ follows a Gaussian distribution with mean and variance in (a) and (b), respectively.
This is because $\hat{w}$ is a linear combination of the random variables $\epsilon_i$, which are Gaussian by assumption.
Since $\hat{w}$ is a linear combination of Gaussian random variables, it is itself Gaussian, with $\hat{w} \sim \mathcal{N}(w, \sigma^2(X^TX)^{-1})$.

\subsection{d: Weighted linear regression}
Suppose we keep the independence assumption but remove the same variance assumption.
In other words, data points would be still sampled independently, but now they may have different variance $\sigma_i$.
Thus, the variance (the covariance matrix) of $\epsilon$ would still be diagonal, but with different values:

$$\Sigma = \begin{bmatrix}
    \sigma_1^2 & 0 & \cdots & 0\\
    0 & \sigma_2^2 & \cdots & 0\\
    \vdots & \vdots & \ddots & \vdots\\
    0 & 0 & \cdots & \sigma_m^2
\end{bmatrix}$$

Derive the estimator $\hat{w}$ (similar to the normal equations) for this problem using matrix-vector notations with $\Sigma$.

\textbf{Answer:}

We want to minimize
$$\arg\min_w \sum_{i=1}^{m} \frac{1}{\sigma_i^2}(y_i - w^T x_i)^2.$$
In matrix-vector notation, this is equivalent to
$$\arg\min_w (Y - Xw)^T \Sigma^{-1} (Y - Xw).$$
Taking the derivative with respect to $w$ and setting it to zero:
\begin{align*}
    \frac{\partial}{\partial w}\left((Y - Xw)^T \Sigma^{-1} (Y - Xw)\right) &= 0\\
    \frac{\partial}{\partial w}\left(w^T X^T \Sigma^{-1} X w - 2w^T X^T \Sigma^{-1} Y + Y^T \Sigma^{-1} Y\right) &= 0\\
    -2X^T \Sigma^{-1} Y + 2X^T \Sigma^{-1} Xw &= 0\\
    (X^T \Sigma^{-1} X)w &= X^T \Sigma^{-1} Y
\end{align*}
Thus, the weighted least squares estimator $\hat{w}$ is:
$$\hat{w} = (X^T \Sigma^{-1} X)^{-1} X^T \Sigma^{-1} Y$$

\section{Ridge Regression}

For linear regression, it is often assumed that $y_i = w^Tx_i + \epsilon$, where $w, x \in \R^d$ by absorbing the constant term (bias) in an affine hypothesis into $w$, and $\epsilon \sim \mathcal{N}(0, \sigma^2)$ is a Gaussian random variable.
Given $m$ i.i.d. samples $z_i = (x_i, y_i), 1 \leq i \leq m$, we define $Y = (y_1, \cdots, y_m)^T$ and $X = (x_1, \cdots, x_m)^T$.
Thus, we have $Y \sim \mathcal{N}(Xw, \sigma^2I_m)$.
Show that the ridge regression estimate is the mean of the posterior distribution under a Gaussian prior $w \sim \mathcal{N}(0, \tau^2I).$
Find the explicit relation between the regularization parameter $\lambda$ in the ridge regression estimate of the parameter $w$, and the variances $\sigma^2, \tau^2$.

\textbf{Answer:}

The ridge regression estimate is defined as
\begin{align*}
    \hat{w}^{\text{Ridge}} &\triangleq \arg\min_w \sum_{i=1}^{m} (w^T x_i - y_i)^2 + \lambda \|w\|^2\\
    &= \arg\min_w \|Xw - Y\|^2 + \lambda\|w\|^2.
\end{align*}
Taking the derivative with respect to $w$ and setting it to zero results in the following expression for $\hat{w}^{\text{Ridge}}$ (as derived in class and in the text):
\begin{align*}
    \hat{w}^{\text{Ridge}} &= (X^T X + \lambda I)^{-1} X^T Y
\end{align*}
Now, we'll show that this estimator is also the mean of the posterior distribution of $w$ when we assume a Gaussian prior $w \sim \mathcal{N}(0, \tau^2I)$.

The posterior distribution of $w$ is proportional to the product of the likelihood and the prior:
\begin{align*}
    p(w|Y) &\propto p(Y|w)p(w)\\
    &\propto \exp\left(-\frac{1}{2\sigma^2}\|Y - Xw\|^2\right)\exp\left(-\frac{1}{2\tau^2}\|w - 0\|^2\right)\\
    &= \exp\left(-\frac{1}{2\sigma^2}\|Xw - Y\|^2\right)\exp\left(-\frac{1}{2\tau^2}\|w\|^2\right)
\end{align*}
\emph{(We neglect the normalization constant $P(Y)$ since it does not depend on $w$.
We also neglect both of the Gaussian normalizing factors since they do not affect the location of the mode.)}

We want to find the mean of this posterior distribution.
Since multiplying two Gaussian PDFs results in another Gaussian PDF, and since the mode of a Gaussian PDF is also its mean, we can find the mean of this posterior by taking its negative and settings its derivative with respect to $w$ to zero.
This value, $\hat{w}^{\text{Mean}}$, will be the (single) maximum of the (Gaussian) posterior distribution:
\begin{align*}
    -log(p(w|Y)) &= -log\left(\exp\left(-\frac{1}{2\sigma^2}\|Xw - Y\|^2\right)\exp\left(-\frac{1}{2\tau^2}\|w\|^2\right)\right)\\
    &= \frac{1}{2\sigma^2}\|Xw - Y\|^2 + \frac{1}{2\tau^2}\|w\|^2\\
    0 &= \dfrac{\partial}{\partial w}\left(\frac{1}{2\sigma^2}\|Xw - Y\|^2 + \frac{1}{2\tau^2}\|w\|^2\right)\\
    0 &= \frac{1}{\sigma^2}X^T(Xw - Y) + \frac{1}{\tau^2}w\\
    0 &= \frac{1}{\sigma^2}X^TXw - \frac{1}{\sigma^2}X^TY + \frac{1}{\tau^2}w\\
    \hat{w}^{\text{Mean}} &= \frac{1}{\sigma^2}\left(\frac{1}{\sigma^2}X^TX + \frac{1}{\tau^2}I\right)^{-1}X^TY
\end{align*}
Now, we can find the value of \(\lambda\) that makes \(\hat{w}^{\text{Mean}}\) equal to \(\hat{w}^{\text{Ridge}}\):
\begin{align*}
    \hat{w}^{\text{Mean}} &= \hat{w}^{\text{Ridge}}\\
    \frac{1}{\sigma^2}\left(\frac{1}{\sigma^2}X^TX + \frac{1}{\tau^2}I\right)^{-1}X^TY &= (X^T X + \lambda I)^{-1} X^T Y\\
    \left(\frac{1}{\sigma^2}X^TX + \frac{1}{\tau^2}I\right)^{-1}\frac{1}{\sigma^2}&= (X^T X + \lambda I)^{-1}\\
    \frac{1}{\sigma^2}&= \left(\frac{1}{\sigma^2}X^TX + \frac{1}{\tau^2}I\right)(X^T X + \lambda I)^{-1}\\
    \frac{1}{\sigma^2} (X^T X + \lambda I)&= \left(\frac{1}{\sigma^2}X^TX + \frac{1}{\tau^2}I\right)\\
    \frac{1}{\sigma^2} X^T X + \frac{\lambda}{\sigma^2} I &= \frac{1}{\sigma^2}X^TX + \frac{1}{\tau^2}I\\
    \frac{\lambda}{\sigma^2} I &= \frac{1}{\tau^2}I\\
    \lambda &= \frac{\sigma^2}{\tau^2}
\end{align*}
Thus, $\hat{w}^{\text{Mean}} = \hat{w}^{\text{Ridge}}$ if we set $\lambda = \frac{\sigma^2}{\tau^2}$.


\section{Lasso estimator}

The LASSO regression problem can be shown to be the following optimization problem:
$$\min_{\vct{w} \in \R^d}\sum_{i=1}^m{(\vct{w}^T\vct{x}_i - y_i)^2} \text{ subject to } \|\vct{w}\|_1 \leq \lambda,$$
where $\lambda > 0$ is a regularization parameter.
Here, we develop a stochastic gradient descent (SDG) algorithm for this problem, which is useful when we have $m >> d$, where $d$ is the dimension of the parameter space.

\subsection{a}
Write $\vct{w} = \vct{w}^{+} - \vct{w}^{-}$, where $\vct{w}^{+}, \vct{w}^{-} \geq 0$ are the positive and negative parts of $\vct{w}$ respectively.
Let $w_j$ be the $j$th component of $\vct{w}$.
When $w_j \leq 0$, $w_j^{+} = 0$ and $w_j^{-} = -w_j$.
Similarly, when $w_j \geq 0$, $w_j^{+} = w_j$ and $w_j^{-} = 0$.
Find a quadratic function, $Q$, of $\vct{w}^{+}$ and $\vct{w}^{-}$ such that
$$\min_{\vct{w}^{+},\vct{w}^{-} \geq 0}\lambda\sum_{i=1}^m{Q(\vct{w}^{+}, \vct{w}^{-})}$$
is equivalent to the above LASSO problem.
Expain the equivalence.

\textbf{Answer:}

[Mohri et al] show in Eqn. 11.33 that the LASSO problem can be rewritten as
$$\min_{\vct{w}^{+},\vct{w}^{-} \geq 0}\sum_{i=1}^{m}\left((\vct{w}^{+} - \vct{w}^{-})\vct{x}_i - y_i\right)^2 + \lambda\sum_{j=1}^d{(w_j^{+} + w_j^{-})}.$$
We can rewrite this in the required quadratic form as follows:
\begin{align*}
    \min_{\vct{w}^{+},\vct{w}^{-} \geq 0}&\lambda\left(\sum_{i=1}^{m}\frac{1}{\lambda}\left((\vct{w}^{+} - \vct{w}^{-})\vct{x}_i - y_i\right)^2 + \sum_{j=1}^d{(w_j^{+} + w_j^{-})}\right)\\
    \min_{\vct{w}^{+},\vct{w}^{-} \geq 0}&\lambda\sum_{i=1}^{m}\left(\frac{1}{\lambda}\left((\vct{w}^{+} - \vct{w}^{-})\vct{x}_i - y_i\right)^2 + \frac{1}{m}\sum_{j=1}^d{(w_j^{+} + w_j^{-})}\right)
\end{align*}
In this form, we can see that
$$Q(\vct{w}^{+}, \vct{w}^{-}) = \frac{1}{\lambda}\left((\vct{w}^{+} - \vct{w}^{-})\vct{x}_i - y_i\right)^2 + \frac{1}{m}\sum_{j=1}^d{(w_j^{+} + w_j^{-})}.$$

\subsection{b}
[Mohri et al Ex. 11.10] Derive a stochastic gradient descent algorithm for the quadratic program (with affine constraints) in part (a).

\subsection{c}
Suppose $X = [x_1, \cdots, x_m]^T$ is orthonormal and there exists a solution $w$ for $Xw = Y$, where $Y = [y_1, \cdots, y_m]^T$ with no more than $k$ non-zero elements.
Can the SGD algorithm get arbitrarily close to $w$?
Explain why or why not.

\section{Logistic Regression}
Logistic regression is named after the log-odds of success (the logit of the probability) defined as below:
$$\ln\left(\frac{P[Y=1|X=x]}{P[Y=0|X=x]}\right),$$
where
$$P[Y=1|X=x] = \frac{\exp(w_0 + w^Tx)}{1 + \exp(w_0 + w^Tx)}.$$

\subsection{a}
Show that log-odds of success is a linear function of $X$.

\subsection{b}
Show that the logistic loss $L(z) = \log(1 + \exp(-z))$ is a convex function.

\section{Programming: Recommendation System}

\emph{Problem Summary:}
A rating by user $u$ on movie $i$ is approximated by

\begin{equation} \label{eq:5}
M_{u,i} \approx \sum_{k=1}^{d}{U_{u,k}V_{i,k}}.
\end{equation}
We want to fit each element of $U$ and $V$ by minimizing squared reconstruction error over all training data points.
That is, the objective function we minimize is given by
\begin{equation} \label{eq:6}
E(U,V) = \sum_{u=1}^n\sum_{i=1}^m{(M_{u,i} - U_uV_v^T)^2} = \sum_{u=1}^n\sum_{i=1}^m{(M_{u,i} - \sum_{k=1}^d{U_{u,k}V_{i,k}})^2},
\end{equation}
where $U_u$ is the $u$th row of $U$ and $V_i$ is the $i$th row of $V$.

We use gradient descent:
\begin{equation} \label{eq:8}
    U_{v,k} \leftarrow U_{v,k} - \mu \frac{\partial E}{\partial U_{v,k}}, \quad V_{j,k} \leftarrow V_{j,k} - \mu \frac{\partial E}{\partial V_{j,k}},
\end{equation}
where $\mu$ is the update rate.

\subsection{a}
Derive the update formula in (\ref{eq:8}) by solving the partial derivatives.

\textbf{Answer:}
\begin{align*}
    \frac{\partial E}{\partial U_{v,k}} &= \frac{\partial}{\partial U_{v,k}}\sum_{u=1}^n\sum_{i=1}^m{(M_{u,i} - \sum_{k'=1}^d{U_{u,k'}V_{i,k'}})^2}\\
    &= \sum_{u=1}^n\sum_{i=1}^m{\frac{\partial}{\partial U_{v,k}}(M_{u,i} - \sum_{k'=1}^d{U_{u,k'}V_{i,k'}})^2}\\
    &= \sum_{u=1}^n\sum_{i=1}^m{2(M_{u,i} - \sum_{k'=1}^d{U_{u,k'}V_{i,k'}})\frac{\partial}{\partial U_{v,k}}(M_{u,i} - \sum_{k'=1}^d{U_{u,k'}V_{i,k'}})}\tag*{\text{Chain rule}}\\
    &= \sum_{u=1}^n\sum_{i=1}^m{2(M_{u,i} - \sum_{k'=1}^d{U_{u,k'}V_{i,k'}})(-\frac{\partial}{\partial U_{v,k}}\sum_{k'=1}^d{U_{u,k'}V_{i,k'}})}\tag*{$\frac{\partial}{\partial U_{v,k}}M_{u,i} = 0$}\\
    &= \sum_{i=1}^m{2(M_{v,i} - \sum_{k'=1}^d{U_{v,k'}V_{i,k'}})(-V_{i,k})}\tag*{$\frac{\partial}{\partial U_{v,k}}U_{u,k'}V_{i,k'} = 0, k' \neq k, u \neq v$}\\
    &= -2\sum_{i=1}^m{(M_{v,i} - \sum_{k'=1}^d{U_{v,k'}V_{i,k'}})V_{i,k}}
\end{align*}
\begin{align*}
    \frac{\partial E}{\partial V_{j,k}} &= \frac{\partial}{\partial V_{j,k}}\sum_{u=1}^n\sum_{i=1}^m{(M_{u,i} - \sum_{k'=1}^d{U_{u,k'}V_{i,k'}})^2}\\
    &= \sum_{u=1}^n\sum_{i=1}^m{2(M_{u,i} - \sum_{k'=1}^d{U_{u,k'}V_{i,k'}})(-\frac{\partial}{\partial V_{j,k}}\sum_{k'=1}^d{U_{u,k'}V_{i,k'}})}\tag*{\text{Same first three steps}}\\
    &= \sum_{u=1}^n{2(M_{u,j} - \sum_{k'=1}^d{U_{u,k'}V_{j,k'}})(-U_{j,k})}\tag*{\text{$\frac{\partial}{\partial V_{j,k}}U_{u,k'}V_{j,k'} = 0, k' \neq k, i \neq j$}}\\
    &= -2\sum_{u=1}^n{(M_{u,j} - \sum_{k'=1}^d{U_{u,k'}V_{j,k'}})U_{j,k}}
\end{align*}

\subsection{b}
Redo part (a) using the regularized objective function below. 
$$E(U,V) = \sum_{u=1}^n\sum_{i=1}^m{(M_{u,i} - \sum_{k=1}^d{U_{u,k}V_{i,k}})^2} + \mu\sum_{u, k}{U_{u,k}^2} + \lambda\sum_{i,k}^m{V_{i,k}^2}$$

\begin{align*}
    \frac{\partial E}{\partial U_{v,k}} &= \frac{\partial}{\partial U_{v,k}}\left(\sum_{u=1}^n\sum_{i=1}^m{(M_{u,i} - \sum_{k'=1}^d{U_{u,k'}V_{i,k'}})^2} + \mu\sum_{u=1}^n\sum_{k'=1}^d{U_{u,k'}^2} + \lambda\sum_{i=1}^m\sum_{k'=1}^d{V_{i,k'}^2}\right)\\
    &= \frac{\partial}{\partial U_{v,k}}\left(\sum_{u=1}^n\sum_{i=1}^m{(M_{u,i} - \sum_{k'=1}^d{U_{u,k'}V_{i,k'}})^2}\right) + \frac{\partial}{\partial U_{v,k}}\mu\sum_{u=1}^n\sum_{k'=1}^d{U_{u,k'}^2} + \frac{\partial}{\partial U_{v,k}}\lambda\sum_{i,k}^m{V_{i,k}^2}\\
    &= -2\sum_{i=1}^m{(M_{v,i} - \sum_{k'=1}^d{U_{v,k'}V_{i,k'}})V_{i,k}} + 2\mu U_{v,k}
\end{align*}
\begin{align*}
    \frac{\partial E}{\partial V_{j,k}} &= \frac{\partial}{\partial V_{j,k}}\left(\sum_{u=1}^n\sum_{i=1}^m{(M_{u,i} - \sum_{k'=1}^d{U_{u,k'}V_{i,k'}})^2}\right) + \frac{\partial}{\partial U_{v,k}}\mu\sum_{u, k}{U_{u,k}^2} + \frac{\partial}{\partial U_{v,k}}\lambda\sum_{i=1}^m\sum_{k'=1}^d{V_{i,k'}^2}\\
    &= -2\sum_{u=1}^n{(M_{u,j} - \sum_{k'=1}^d{U_{u,k'}V_{j,k'}})U_{j,k}} + 2\lambda V_{j,k}
\end{align*}

\end{document}

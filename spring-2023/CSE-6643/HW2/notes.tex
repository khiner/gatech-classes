\documentclass[twoside,10pt]{article}
%=================================================
% Basics
%=================================================
\usepackage{fixltx2e} % Makes \( \) equation style robust, among other
                      % things. Must be the first package.


% Makes ligatured fonts searchable and copyable in pdf readers
\usepackage{cmap} % Load before fontenc 

% Always include these font encodings in your document 
% unless you have a very good reason.
\usepackage[T1]{fontenc}
\usepackage[utf8]{inputenc}

\usepackage{verbatim}

%=============
% Fonts
%=============

\usepackage{lmodern} % Improved version of computer modern
\usepackage[scale=0.88]{tgheros} % Helvetica clone for sans serif font


\newcommand\hmmax{2} % Default is 3.
\newcommand\bmmax{2} % Default is 4.

\usepackage{bm} % boldmath must be called after the package
\providecommand{\mathbold}[1]{\bm{#1}}

%=============
% AMS Packages and fonts
%=============
\usepackage{amsmath,amsbsy,amsgen,amscd,amsthm,amsfonts,amssymb} 

%=============
% Margins and paper size
%=============
\usepackage[centering,top=1.5in,bottom=1.2in,left=1.4in,right=1.4in]{geometry}
\usepackage{parskip}


%=============
% Section headings
%=============
\usepackage[sf,bf,compact]{titlesec}

%=============
% Tables and lists
%=============
\usepackage{booktabs,longtable,tabu} % Nice tables
\setlength{\tabulinesep}{1mm}
\usepackage[font=small,margin=10pt,labelfont={sf,bf},labelsep={space}]{caption}

%=============
% Code output
%=============
% \usepackage{listings}
% \usepackage{minted}




\usepackage{enumitem}
\setitemize{itemsep=0pt} 
\setenumerate{itemsep=0pt}
\setlist{labelindent=\parindent,%  % Recommended by enumitem package
  font=\sffamily}


%=============
% Hyperlink colors
%=============
\usepackage[usenames,dvipsnames]{xcolor}
\definecolor{steelblue}{HTML}{A1BDC7}
\definecolor{orange}{HTML}{D98C21}
\definecolor{silver}{HTML}{B0ABA8}
\definecolor{rust}{HTML}{B8420F}
\definecolor{seagreen}{HTML}{2E6B69}
\definecolor{joshua}{HTML}{FBDC7F}
\definecolor{darksky}{HTML}{154c79}

\colorlet{steelblue}{silver!30!white}
\colorlet{darkorange}{orange!85!black}
\colorlet{darksilver}{silver!85!black}
\colorlet{darksteelblue}{steelblue!85!black}
\colorlet{darkrust}{rust!85!black}
\colorlet{darkseagreen}{seagreen!85!black}

\usepackage{url}
\usepackage[colorlinks=true]{hyperref}
\hypersetup{linkcolor=darkrust}    
\hypersetup{citecolor=darkseagreen}      
\hypersetup{urlcolor=darksilver}     

%=============
% Microtype
%=============
\usepackage[final]{microtype} 

%=====================
% Header
%=====================
% \usepackage{fancyhdr}
% \usepackage{nopageno} % Gets rid of page number at the bottom
% \fancyhf{} % Clear header style
% \renewcommand{\headrulewidth}{0.5pt} % remove the header rule
% \pagestyle{fancy}
% \fancyhead[LE,RO]{\textsf{\small \thepage}}
% 
% \setlength{\headheight}{14pt}
%=====================
% Fix delimiters
%=====================

% Fixes \left and \right spacing issues. See discussion at
% http://tex.stackexchange.com/questions/2607/spacing-around-left-and-right
\let\originalleft\left
\let\originalright\right
\renewcommand{\left}{\mathopen{}\mathclose\bgroup\originalleft}
\renewcommand{\right}{\aftergroup\egroup\originalright}

%=================================================
% Math macros
%=================================================

%=============
% Generalities
%=============
\usepackage{mathtools}
\mathtoolsset{centercolon}  % Makes := typeset correctly for definitions

%%% Equation numbering
%\numberwithin{equation}{section} 

%%% Annotations
\newcommand{\notate}[1]{\textcolor{red}{\textbf{[#1]}}}

%==============
% Symbols
%==============
\let\oldphi\phi
\let\oldeps\epsilon

\renewcommand{\phi}{\varphi}
\renewcommand{\epsilon}{\varepsilon}
\newcommand{\eps}{\varepsilon}

%==============
% Constants
%==============

% Set constants upright
\newcommand{\cnst}[1]{\mathrm{#1}}  
\newcommand{\econst}{\mathrm{e}}

\newcommand{\zerovct}{\vct{0}} % Zero vector
\newcommand{\Id}{\mathbf{I}} % Identity matrix
\newcommand{\onemtx}{\bm{1}}
\newcommand{\zeromtx}{\bm{0}}

%==============
% Sets
%==============
\providecommand{\mathbbm}{\mathbb} % In case we don't load bbm

% Reals, complex, naturals
\newcommand{\R}{\mathbbm{R}}
\newcommand{\C}{\mathbbm{C}}
\newcommand{\K}{\mathbbm{K}}
\newcommand{\N}{\mathbbm{N}}

%==============
% Probability
%==============
\newcommand{\Prob}{\operatorname{\mathbbm{P}}}
\newcommand{\Expect}{\operatorname{\mathbb{E}}}

%==============
% Vectors and matrices 
%==============
\newcommand{\vct}[1]{\mathbold{#1}}
\newcommand{\mtx}[1]{\mathbold{#1}}

\newcommand{\mrange}{\operatorname{range}}
\newcommand{\mnull}{\operatorname{null}}



\begin{document}

For any symmetric matrix, $\mtx{S} = \mtx{S}^T$.

For any real matrix, $\mtx{A}^T\mtx{A}$ is symmetric.

$\sigma_i(\mtx{A}) = \sqrt{|\lambda_i(\mtx{A}^T\mtx{A})|}$

For any vector, $\|\vct{x}\|^2 = \vct{x}^T\vct{x} = \vct{x} \cdot \vct{x}$.

Triangle inequality (works for both vector and matrix norms): $\|\vct{x} + \vct{y}\| \leq \|\vct{x}\| + \|\vct{y}\|$

Cauchy-Schwarz: $|\vct{u} \cdot \vct{v}| \leq \|\vct{u}\| \|\vct{v}\|$, or $|\vct{u} \cdot \vct{v}|^2 \leq (\vct{u}\cdot\vct{u})(\vct{v}\cdot\vct{v})$

Matrix norms are submultiplicitive: $\|\mtx{A}\mtx{B}\| \leq \|\mtx{A}\|\|\mtx{B}\|$

Eigenvectors of a symmetric matrix, or an orthogonal matrix, are orthogonal.

For orthogonal matrix $\mtx{Q}$, $\mtx{Q}^T\mtx{Q} = \mtx{I}$.
If $\mtx{Q}$ is square, then $\mtx{Q}\mtx{Q}^T = \mtx{I}$.

The eigenvectors of a real symmetric matrix are real.
(But an orthogonal matrix may have complex eigenvectors.)
Furthermore, the eigenvalues of a symmetric matrix are real and can be ordered so that $\lambda_1 \geq \lambda_2 \geq ... \geq \lambda_n$. 

Similar matrices: If $\mtx{B} = \mtx{M}^{-1}\mtx{A}\mtx{M}$, then $\mtx{B}$ and $\mtx{A}$ have the same eigenvalues.

Schur's triangularization theorem: Any square matrix can be transformed into an upper triangular matrix by a similarity transformation.

Orthogonality of eigenvectors: If $\lambda_1$ and $\lambda_2$ are distinct eigenvalues of $\mtx{A}$, and $\vct{v_1}$ and $\vct{v_2}$ are their corresponding eigenvectors, then $\vct{v_1}$ and $\vct{v_2}$ are orthogonal.

Singular value inequality: The singular values of a matrix are always non-negative and satisfy the inequality $\sigma_1 \geq \sigma_2 \geq ... \geq \sigma_n$.

Positive definite matrix: A symmetric matrix $\mtx{A}$ is positive definite if $\vct{v}^T\mtx{A}\vct{v} > 0$ for all non-zero vectors $\vct{v}$.

$\mtx{A}\mtx{B}$ has the same non-zero eigenvalues as $\mtx{B}\mtx{A}.$

Trace: Adding the diagonals of a matrix gives you the sum of the eigenvalues.

Multiply eigenvalues, or singular values, to get determinant.

$\mtx{A}=\mtx{X}\mtx{\Lambda}\mtx{X}$, for any matrix, where the columns of $\mtx{X}$ are the eigenvectors of $\mtx{A}$ and $\mtx{\Lambda}$ is the diagonal matrix of eigenvalues.

If $\mtx{S}, \mtx{T}$ both positive definite, $\mtx{S} + \mtx{T}$ pos. def. as well. And also $\mtx{S}^{-1}$ (since its eigenvalues are $\frac{1}{\lambda}$, so all positive).

If rank is $n$, only $n$ nonzero eigenvalues.

For unitary (orthonormal) matrix $\mtx{Q}$, $(\mtx{Q}\vct{x})^{*}(\mtx{Q}\vct{x}) = x^{*}y$ and $\|\mtx{Q}\vct{x}\|_2 = \|\vct{x}\|_2$.

If $\mtx{D}$ is diagonal, then $\|\mtx{D}\|_P = \max_1\leq i \leq m{|d_i|}$.

  \begin{equation*}
    \sigma_{\mathrm{max}}(\mtx{A}) = \|\mtx{A}\|_2 = \sqrt{|\lambda_{\mathrm{max}}(\mtx{A}^T\mtx{A})|} = \sup_{x \neq 0, \|x\|_2 = 1}{\|\mtx{A}\vct{x}\|_2}
  \end{equation*}
  
\href{https://en.wikipedia.org/wiki/Matrix_norm#Matrix_norms_induced_by_vector_norms}{Matrix norms wikipedia}

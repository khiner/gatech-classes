\documentclass[twoside,10pt]{article}
%=================================================
% Basics
%=================================================
\usepackage{fixltx2e} % Makes \( \) equation style robust, among other
                      % things. Must be the first package.


% Makes ligatured fonts searchable and copyable in pdf readers
\usepackage{cmap} % Load before fontenc 

% Always include these font encodings in your document 
% unless you have a very good reason.
\usepackage[T1]{fontenc}
\usepackage[utf8]{inputenc}

\usepackage{verbatim}

%=============
% Fonts
%=============

\usepackage{lmodern} % Improved version of computer modern
\usepackage[scale=0.88]{tgheros} % Helvetica clone for sans serif font


\newcommand\hmmax{2} % Default is 3.
\newcommand\bmmax{2} % Default is 4.

\usepackage{bm} % boldmath must be called after the package
\providecommand{\mathbold}[1]{\bm{#1}}

%=============
% AMS Packages and fonts
%=============
\usepackage{amsmath,amsbsy,amsgen,amscd,amsthm,amsfonts,amssymb} 

%=============
% Margins and paper size
%=============
\usepackage[centering,top=1.5in,bottom=1.2in,left=1.4in,right=1.4in]{geometry}
\usepackage{parskip}


%=============
% Section headings
%=============
\usepackage[sf,bf,compact]{titlesec}

%=============
% Tables and lists
%=============
\usepackage{booktabs,longtable,tabu} % Nice tables
\setlength{\tabulinesep}{1mm}
\usepackage[font=small,margin=10pt,labelfont={sf,bf},labelsep={space}]{caption}

%=============
% Code output
%=============
% \usepackage{listings}
% \usepackage{minted}




\usepackage{enumitem}
\setitemize{itemsep=0pt} 
\setenumerate{itemsep=0pt}
\setlist{labelindent=\parindent,%  % Recommended by enumitem package
  font=\sffamily}


%=============
% Hyperlink colors
%=============
\usepackage[usenames,dvipsnames]{xcolor}
\definecolor{steelblue}{HTML}{A1BDC7}
\definecolor{orange}{HTML}{D98C21}
\definecolor{silver}{HTML}{B0ABA8}
\definecolor{rust}{HTML}{B8420F}
\definecolor{seagreen}{HTML}{2E6B69}
\definecolor{joshua}{HTML}{FBDC7F}
\definecolor{darksky}{HTML}{154c79}

\colorlet{steelblue}{silver!30!white}
\colorlet{darkorange}{orange!85!black}
\colorlet{darksilver}{silver!85!black}
\colorlet{darksteelblue}{steelblue!85!black}
\colorlet{darkrust}{rust!85!black}
\colorlet{darkseagreen}{seagreen!85!black}

\usepackage{url}
\usepackage[colorlinks=true]{hyperref}
\hypersetup{linkcolor=darkrust}    
\hypersetup{citecolor=darkseagreen}      
\hypersetup{urlcolor=darksilver}     

%=============
% Microtype
%=============
\usepackage[final]{microtype} 

%=====================
% Header
%=====================
% \usepackage{fancyhdr}
% \usepackage{nopageno} % Gets rid of page number at the bottom
% \fancyhf{} % Clear header style
% \renewcommand{\headrulewidth}{0.5pt} % remove the header rule
% \pagestyle{fancy}
% \fancyhead[LE,RO]{\textsf{\small \thepage}}
% 
% \setlength{\headheight}{14pt}
%=====================
% Fix delimiters
%=====================

% Fixes \left and \right spacing issues. See discussion at
% http://tex.stackexchange.com/questions/2607/spacing-around-left-and-right
\let\originalleft\left
\let\originalright\right
\renewcommand{\left}{\mathopen{}\mathclose\bgroup\originalleft}
\renewcommand{\right}{\aftergroup\egroup\originalright}

%=================================================
% Math macros
%=================================================

%=============
% Generalities
%=============
\usepackage{mathtools}
\mathtoolsset{centercolon}  % Makes := typeset correctly for definitions

%%% Equation numbering
%\numberwithin{equation}{section} 

%%% Annotations
\newcommand{\notate}[1]{\textcolor{red}{\textbf{[#1]}}}

%==============
% Symbols
%==============
\let\oldphi\phi
\let\oldeps\epsilon

\renewcommand{\phi}{\varphi}
\renewcommand{\epsilon}{\varepsilon}
\newcommand{\eps}{\varepsilon}

%==============
% Constants
%==============

% Set constants upright
\newcommand{\cnst}[1]{\mathrm{#1}}  
\newcommand{\econst}{\mathrm{e}}

\newcommand{\zerovct}{\vct{0}} % Zero vector
\newcommand{\Id}{\mathbf{I}} % Identity matrix
\newcommand{\onemtx}{\bm{1}}
\newcommand{\zeromtx}{\bm{0}}

%==============
% Sets
%==============
\providecommand{\mathbbm}{\mathbb} % In case we don't load bbm

% Reals, complex, naturals
\newcommand{\R}{\mathbbm{R}}
\newcommand{\C}{\mathbbm{C}}
\newcommand{\K}{\mathbbm{K}}
\newcommand{\N}{\mathbbm{N}}

%==============
% Probability
%==============
\newcommand{\Prob}{\operatorname{\mathbbm{P}}}
\newcommand{\Expect}{\operatorname{\mathbb{E}}}

%==============
% Vectors and matrices 
%==============
\newcommand{\vct}[1]{\mathbold{#1}}
\newcommand{\mtx}[1]{\mathbold{#1}}

\newcommand{\mrange}{\operatorname{range}}
\newcommand{\mnull}{\operatorname{null}}



\begin{document}

\title{CSE 6643 Homework 7}
\author{Karl Hiner, Spring 2023}
\date{}
\maketitle

\section{Convergence of Lanczos iteration [60 pts]}
Consider the Lanczos process for a symmetric matrix $\mtx{A} \in \R^{m \times m}$, with eigenvalues $\lambda_{1} \geq \lambda_{2} \geq \dots \geq \lambda_{m}$.  
We denote by $\mtx{T}_k$ the $k \times k$ tri-diagonal matrix computed by the Lancczos process starting with the vector $\vct{q}_1$. 
Recall that
\begin{equation}
  \mtx{T}_k = \mtx{Q}^T_k \mtx{A} \mtx{Q}_k.
\end{equation}
The largest eigenvalue $\mu_k$ of $\mtx{T}_k$ satisfies
\begin{equation}
  \mu_{k} = \max \limits_{\vct{y} \neq \vct{0}} \frac{\vct{y}^T \mtx{T}_k \vct{y}}{\vct{y}^T \vct{y}}.
\end{equation}
You can use (or prove, for bonus points, in problem 3) that 
\begin{equation}
  \mu_{k} \leq \lambda_1.
\end{equation}
In this exercise you will also need the Chebychev polynomials $T_k(x)$ satisfying  
\begin{align}
  T_k\left(\cos \theta\right) &= \cos\left(k \theta \right),\\
  T_k\left(\cosh y\right) &= \cosh\left(k y \right).
\end{align}
Using the Chebychev polynomials, we define 
\begin{equation}
  p_k(x) = T_{k - 1} \left( - 1 + 2 \frac{x - \lambda_m}{\lambda_2 - \lambda_m}\right).
\end{equation}

\subsection*{(a) [10 pts]}
Consider a polynomial $p(x)$ of degree $k - 1$.
Show that
\begin{equation}
  \mu_k \geq \frac{\vct{q}_1^T p(\mtx{A}) \mtx{A} p(\mtx{A}) \vct{q}_1}{\vct{q}_1^T p(\mtx{A})^2 \vct{q}_1}
\end{equation}

\begin{align*}
  \mu_k &= \max\limits_{\vct{y} \neq 0}\dfrac{\vct{y}^T\mtx{T}_k\vct{y}}{\vct{y}^T\vct{y}}&\text{(given)}\\
  &= \max\limits_{\vct{y} \neq 0}\dfrac{\vct{y}^T\mtx{Q}_k^T \mtx{A}\mtx{Q}_k\vct{y}}{\vct{y}^T\vct{y}}&\text{(substitute $\mtx{T}_k = \mtx{Q}^T_k \mtx{A} \mtx{Q}_k$)}\\
  &= \max\limits_{\vct{y} \neq 0}\dfrac{\left(\mtx{Q_k}\vct{y}\right)^T \mtx{A}\left(\mtx{Q}_k\vct{y}\right)}{\vct{y}^T\vct{y}}&\text{(rearrange numerator)}\\
  &= \max\limits_{\vct{x} \in \mathcal{K}(A,q,k) \setminus \emptyset}\dfrac{\vct{x}^T \mtx{A}\vct{x}}{\vct{x}^T\vct{x}}&\text{($\vct{x} \coloneqq \mtx{Q}_k\vct{y}$, $\mathcal{K}$ is the Krylov subspace of $\mtx{A}$)}\\
\end{align*}

The last step follows since $\mtx{Q}_k$ is obtained by applying the Gram-Schmidt process to the Krylov subspace $\mathcal{K}(A,q,k)$, and thus $\mtx{Q}_k$ is an orthonormal basis for $\mathcal{K}(A,q,k)$.

Since $\vct{x}$ is in the Krylov space of $\mtx{A}$, we can (by the definition of the Krylov space) express it as $\vct{x} = p(\mtx{A})\vct{b}$, where $p$ is a polynomial of degree $k - 1$, and we have
$$\mu_k = \max \limits_{p \text{ of deg } k - 1} \frac{\vct{b}^T p(\mtx{A})^T\mtx{A} p(\mtx{A})\vct{b}}{\vct{b}^T f(\mtx{A})^2\vct{b}}.$$

Since the starting vector $\vct{q}_1$ is arbitrary, we can choose $\vct{b} = \vct{q}_1$, which completes the proof.

\subsection*{(b) [10 pts]}
Expand $\vct{q}_1$ in a basis of orthonormal eigenvectors $\left\{\vct{z}_i\right\}_{1 \leq i \leq m}$ of $\mtx{A}$.
\begin{equation}
  \vct{q}_1 = y_1 \vct{z}_1 + \dots + y_m \vct{z}_m. 
\end{equation}
Prove that 
\begin{equation}
  \frac{\vct{q}_1^T p(\mtx{A}) \mtx{A} p(\mtx{A}) \vct{q}_1}{\vct{q}_1^T p(\mtx{A})^2 \vct{q}_1} = \frac{\sum \limits_{i = 1}^{m} y_{i}^2 p(\lambda_i)^2 \lambda_{i}}{\sum \limits_{i = 1}^{m} y_{i}^2 p(\lambda_i)^2}
\end{equation}

\subsection*{(c) [10 pts]}
Show that $|p_k(\lambda_i)| \leq 1$, $\forall$ $2 \leq i \leq m$. 

\begin{align*}
  \lambda_m &\leq \lambda_i \leq \lambda_2&\text{(for $2 \leq i \leq m$, since $\lambda_{1} \geq \lambda_{2} \geq \lambda_{m}$)}\\
  0 &\leq \lambda_i - \lambda_m \leq \lambda_2 - \lambda_m&\text{(subtract $\lambda_m$)}\\
  0 &\leq \frac{\lambda_i - \lambda_m}{\lambda_2 - \lambda_m} \leq 1&\text{(divide by $\lambda_2 - \lambda_m$)}\\
  -1 &\leq -1 + 2 \frac{\lambda_i - \lambda_m}{\lambda_2 - \lambda_m} \leq 1 &\text{(multiply by 2 and subtract 1)}\\
  -1 &\leq x_i \leq 1 \implies |x_i| \leq 1&\text{($x_i \coloneqq - 1 + 2 \dfrac{\lambda_i - \lambda_m}{\lambda_2 - \lambda_m}$)}
\end{align*}
Given the definition of $x_i$, note that $p_k(\lambda_i) = T_{k - 1}(x_i)$.
Thus, we want to show that $|x_i| \leq 1$ implies $|T_{k - 1}(x_i)| \leq 1$.

We are given that $T_k\left(\cos \theta\right) = \cos\left(k \theta \right)$.
Since $\cos \theta \in [-1, 1], \forall \theta$, and since $\cos \theta$ produces all values in the range $[-1, 1]$, we can deduce that for any value of $x_i$, $|T_{k - 1}(x_i)| \leq 1$.

Thus, for $2 \leq i \leq m$, we have shown that $|p_k(\lambda_i)| = |T_{k - 1}(x_i)| \leq 1$.

\subsection*{(d) [10 pts]}
Show that
\begin{equation}
  p_k(\lambda_1) \in O\left[ \left(2 \left(2 \frac{\lambda_1 - \lambda_2}{\lambda_2 - \lambda_m} + 1\right)\right)^{k - 1} \right],
\end{equation}
in the limit of $(\lambda_{1} - \lambda_{2}) / (\lambda_{2} - \lambda_{m}) \rightarrow \infty$. 

\subsection*{(e) [10 pts]}
Show that with $\cos{\phi_1} = \vct{q}_1^T \vct{z}_1$, 
\begin{equation}
  \lambda_{1} - (\lambda_{1} - \lambda_{m}) \frac{(\tan (\phi_1))^2}{p_{k}(\lambda_1)^2} \leq \mu_{k} \leq \lambda_1.
\end{equation}

\subsection*{(f) [10 pts]}

Assume, for example, that $\lambda_1 - \lambda_2 = (\lambda_2 - \lambda_m) / 2.$  
Find $z$ such that 
\begin{equation}
  p_k (\lambda_1) \approx \frac{z^{k - 1}}{2}.
\end{equation}
Estimate the number of iterations to reach machine accuracy $u$ for $\lambda_1$. 

\section{Arnoldi vs Lanczos [40 pts]}
\subsection*{Arnoldi [15 pts]} 
Go to section (a) of the file \texttt{HW7\_your\_code.jl} and implement a function that implements the Arnoldi method. 
Your method should return an upper Hessenberg matrix. 

\subsection*{(b) Lanczos [15 pts]} 
Go to section (b) of the file \texttt{HW7\_your\_code.jl} and implement a function that implements the Lanczos method. 
Your method should return an upper Hessenberg matrix. 

\subsection*{(c) Arnoldi vs Lanczos [10 pts]}
Compare the behavior of the Arnoldi and Lanczos method on different symmetric matrices. 
How does the approximation improve as you increase $k_{\max}$? 
Can you create reproduce scenarios where Lanczos methods fails, for instance by producing ghost eigenvalues?
Summarize your findings using figures as appropriate, and hand them in \underline{with the written part of your homework}. 

\section{Nested Approximation [20 bonus pts]}
Let $\mtx{A} \in \R^{m \times m}$ be symmetric, and $\mtx{Q} \in \R^{m \times m}$ be orthogonal with columns $\vct{q}_i$. 
Define $\mtx{Q}_k \coloneqq \mtx{Q}_{:, 1 : k}$. 
We denote 
\begin{align}
  \overline{\mu}_k &\coloneqq \lambda_{\max} \left(\mtx{Q}_k^T \mtx{A} \mtx{Q}_k\right),\\
  \underline{\mu}_k &\coloneqq \lambda_{\min}\left(\mtx{Q}^T_k \mtx{A} \mtx{Q}_k\right),
\end{align}
where $\lambda_{\max}$ and $\lambda_{\min}$ are the largest and smallest eigenvalues.

Prove that
\begin{align}
  \overline{\mu}_{1} \leq \overline{\mu}_{2} \leq \dots \leq \overline{\mu}_m &= \lambda_{\max} \left(\mtx{A}\right),\\
  \underline{\mu}_{1} \geq \underline{\mu}_{2} \geq \dots \geq \underline{\mu}_m &= \lambda_{\min} \left(\mtx{A}\right).
\end{align}

\begin{align*}
  \overline{\mu}_k &\coloneqq \lambda_{\max} \left(\mtx{Q}_k^T \mtx{A} \mtx{Q}_k\right)&\text{(given)}\\
  &= \max\limits_{\vct{y} \neq 0}\dfrac{\vct{y}^T\mtx{Q}_k^T \mtx{A}\mtx{Q}_k\vct{y}}{\vct{y}^T\vct{y}}&\text{(by the Courant-Fischer theorem)}\\
  &= \max\limits_{\vct{y} \neq 0}\dfrac{\left(\mtx{Q_k}\vct{y}\right)^T \mtx{A}\left(\mtx{Q}_k\vct{y}\right)}{\vct{y}^T\vct{y}}&\text{(rearrange numerator)}\\
  &= \max\limits_{\vct{x} \in \mathcal{K}(A,q,k) \setminus \emptyset}\dfrac{\vct{x}^T \mtx{A}\vct{x}}{\vct{x}^T\vct{x}}&\text{($\vct{x} \coloneqq \mtx{Q}_k\vct{y}$, $\mathcal{K}$ is the Krylov subspace of $\mtx{A}$)}\\
\end{align*}

The last step follows since $\mtx{Q}_k$ is obtained by applying the Gram-Schmidt process to the Krylov subspace $\mathcal{K}(A,q,k)$, and thus $\mtx{Q}_k$ is an orthonormal basis for $\mathcal{K}(A,q,k)$.

By the Courant-Fischer theorem, we can also express largest eigenvalue of $\mtx{A}$ as
$$\lambda_{\max} \left(\mtx{A}\right) = \max\limits_{\vct{x} \neq 0}\dfrac{\vct{x}^T\mtx{A}\vct{x}}{\vct{x}^T\vct{x}}.$$

Since $\overline{\mu}_k$ is found by maximizing the same objective as $\lambda_{\max} \left(\mtx{A}\right)$, but over a smaller subset, we have $\overline{\mu}_k \leq \lambda_{\max} \left(\mtx{A}\right)$.

\section{Early Termination [20 bonus pts]}
Let $\mtx{A} \in \R^{m \times m}$ be a symmetric matrix. 
We denote by $K$ the dimension of the Krylov subspace 
\begin{equation}
  \mathcal{K}\left(\mtx{A}, \vct{q}_1, m\right) = \left\{\vct{q}_1, \mtx{A}\vct{q}_1, \mtx{A}^2 \vct{q}_1, \ldots, \mtx{A}^{m - 1} \vct{q}_1\right\}.
\end{equation}
The Lanczos algorithm is said to terminate after $k$ steps if $\beta_{k} = 0$ and $\beta_{i} \neq 0, \forall i < k$.

Prove that the Lanczos process, with starting vectors $\vct{q}_1$, terminates in at most $K$ steps in exact arithmetic. 

\section{CIOS [10 bonus pts for the entire class]}
  In order to help the TAs and myself in improving the course, please make sure to fill out your CIOS. 
  I take this feedback into account for future iterations of this class, so please aim to provide me with feedback both on what needs improvement and what is good as is. 
  If $\geq 75$ percent of students in this class fill out their CIOS by the time the homework is graded, everyone in the class earns 10 bonus points.
\end{document}
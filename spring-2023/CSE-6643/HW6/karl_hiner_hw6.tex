\documentclass[twoside,10pt]{article}
%=================================================
% Basics
%=================================================
\usepackage{fixltx2e} % Makes \( \) equation style robust, among other
                      % things. Must be the first package.


% Makes ligatured fonts searchable and copyable in pdf readers
\usepackage{cmap} % Load before fontenc 

% Always include these font encodings in your document 
% unless you have a very good reason.
\usepackage[T1]{fontenc}
\usepackage[utf8]{inputenc}

\usepackage{verbatim}

%=============
% Fonts
%=============

\usepackage{lmodern} % Improved version of computer modern
\usepackage[scale=0.88]{tgheros} % Helvetica clone for sans serif font


\newcommand\hmmax{2} % Default is 3.
\newcommand\bmmax{2} % Default is 4.

\usepackage{bm} % boldmath must be called after the package
\providecommand{\mathbold}[1]{\bm{#1}}

%=============
% AMS Packages and fonts
%=============
\usepackage{amsmath,amsbsy,amsgen,amscd,amsthm,amsfonts,amssymb} 

%=============
% Margins and paper size
%=============
\usepackage[centering,top=1.5in,bottom=1.2in,left=1.4in,right=1.4in]{geometry}
\usepackage{parskip}


%=============
% Section headings
%=============
\usepackage[sf,bf,compact]{titlesec}

%=============
% Tables and lists
%=============
\usepackage{booktabs,longtable,tabu} % Nice tables
\setlength{\tabulinesep}{1mm}
\usepackage[font=small,margin=10pt,labelfont={sf,bf},labelsep={space}]{caption}

%=============
% Code output
%=============
% \usepackage{listings}
% \usepackage{minted}




\usepackage{enumitem}
\setitemize{itemsep=0pt} 
\setenumerate{itemsep=0pt}
\setlist{labelindent=\parindent,%  % Recommended by enumitem package
  font=\sffamily}


%=============
% Hyperlink colors
%=============
\usepackage[usenames,dvipsnames]{xcolor}
\definecolor{steelblue}{HTML}{A1BDC7}
\definecolor{orange}{HTML}{D98C21}
\definecolor{silver}{HTML}{B0ABA8}
\definecolor{rust}{HTML}{B8420F}
\definecolor{seagreen}{HTML}{2E6B69}
\definecolor{joshua}{HTML}{FBDC7F}
\definecolor{darksky}{HTML}{154c79}

\colorlet{steelblue}{silver!30!white}
\colorlet{darkorange}{orange!85!black}
\colorlet{darksilver}{silver!85!black}
\colorlet{darksteelblue}{steelblue!85!black}
\colorlet{darkrust}{rust!85!black}
\colorlet{darkseagreen}{seagreen!85!black}

\usepackage{url}
\usepackage[colorlinks=true]{hyperref}
\hypersetup{linkcolor=darkrust}    
\hypersetup{citecolor=darkseagreen}      
\hypersetup{urlcolor=darksilver}     

%=============
% Microtype
%=============
\usepackage[final]{microtype} 

%=====================
% Header
%=====================
% \usepackage{fancyhdr}
% \usepackage{nopageno} % Gets rid of page number at the bottom
% \fancyhf{} % Clear header style
% \renewcommand{\headrulewidth}{0.5pt} % remove the header rule
% \pagestyle{fancy}
% \fancyhead[LE,RO]{\textsf{\small \thepage}}
% 
% \setlength{\headheight}{14pt}
%=====================
% Fix delimiters
%=====================

% Fixes \left and \right spacing issues. See discussion at
% http://tex.stackexchange.com/questions/2607/spacing-around-left-and-right
\let\originalleft\left
\let\originalright\right
\renewcommand{\left}{\mathopen{}\mathclose\bgroup\originalleft}
\renewcommand{\right}{\aftergroup\egroup\originalright}

%=================================================
% Math macros
%=================================================

%=============
% Generalities
%=============
\usepackage{mathtools}
\mathtoolsset{centercolon}  % Makes := typeset correctly for definitions

%%% Equation numbering
%\numberwithin{equation}{section} 

%%% Annotations
\newcommand{\notate}[1]{\textcolor{red}{\textbf{[#1]}}}

%==============
% Symbols
%==============
\let\oldphi\phi
\let\oldeps\epsilon

\renewcommand{\phi}{\varphi}
\renewcommand{\epsilon}{\varepsilon}
\newcommand{\eps}{\varepsilon}

%==============
% Constants
%==============

% Set constants upright
\newcommand{\cnst}[1]{\mathrm{#1}}  
\newcommand{\econst}{\mathrm{e}}

\newcommand{\zerovct}{\vct{0}} % Zero vector
\newcommand{\Id}{\mathbf{I}} % Identity matrix
\newcommand{\onemtx}{\bm{1}}
\newcommand{\zeromtx}{\bm{0}}

%==============
% Sets
%==============
\providecommand{\mathbbm}{\mathbb} % In case we don't load bbm

% Reals, complex, naturals
\newcommand{\R}{\mathbbm{R}}
\newcommand{\C}{\mathbbm{C}}
\newcommand{\K}{\mathbbm{K}}
\newcommand{\N}{\mathbbm{N}}

%==============
% Probability
%==============
\newcommand{\Prob}{\operatorname{\mathbbm{P}}}
\newcommand{\Expect}{\operatorname{\mathbb{E}}}

%==============
% Vectors and matrices 
%==============
\newcommand{\vct}[1]{\mathbold{#1}}
\newcommand{\mtx}[1]{\mathbold{#1}}

\newcommand{\mrange}{\operatorname{range}}
\newcommand{\mnull}{\operatorname{null}}



\begin{document}

\title{CSE 6643 Homework 6}
\author{Karl Hiner, Spring 2023}
\date{}
\maketitle

\section{Convergence of QR iteration [50 pts]}
  In this problem, we consider the convergence rate of the QR algorithm with a single-shift strategy. We consider a real matrix $\mtx{A} \in \R^{m \times m}$. 
  The QR iteration can be written as follows:
  \begin{align}
    \mtx{A}^{(0)} &= \mtx{A} \\
    \mtx{A}^{(k)} &= \mu_k \Id + \mtx{Q}_k \mtx{R}_k,\\
    \mtx{A}^{(k+1)} &= \mtx{R}_k \mtx{Q}_k + \mu_k \Id.
  \end{align}
  If we choose $\mu_k = \mtx{A}_{m, m}^{(k)}$ to be the bottom-right entry of the matrix $\mtx{A}^{(k)}$, then this is called the \emph{single-shift QR iteration.}

  Prove the following results.
  You may use figures to illustrate your explanations. 
  \subsection*{(a) [10 pts]}
  Show that if $\mtx{A}^{(0)} = \mtx{A}$ is an upper Hessenberg matrix, then $\mtx{A}^{(k)}$ is upper Hessenberg for all $k \geq 0$.
  Thus, from now on, we always assume that the matrix $\mtx{A}$ is an upper Hessenberg matrix.

    \quad We use induction. For $k = 0$, $\mtx{A}^{(0)} = \mtx{A}$ is upper Hessenberg by assumption.
    
    Now, assume that $\mtx{A}^{(k)}$ is upper Hessenberg.
    To find $\mtx{A}^{(k+1)}$, we perform a QR factorization of $\mtx{A}^{(k)} - \mu_k \Id $.
    
    Since $\mtx{A}^{(k)} - \mu_k \Id$ is an upper Hessenberg matrix, the resulting $\mtx{Q}_k$ factor will also be upper Hessenberg.

    To show this, multiply both sides of $\mtx{Q}_k \mtx{R}_k = \mtx{A}^{(k)} - \mu_k \Id$ by $\mtx{R}_k^{-1}$ to get
    $$\mtx{Q}_k = (\mtx{A}^{(k)} - \mu_k \Id) \mtx{R}_k^{-1}.$$
    
    $\mtx{A}^{(k)}$ is upper Hessenberg by our inductive assumption.
    $\mu_k \Id$ is a diagonal matrix, so $\mtx{A}^{(k)} - \mu_k \Id$ is also upper Hessenberg.
    $\mtx{R}_k$ is an upper triangular matrix, so $\mtx{R}_k^{-1}$ is also upper triangular.
  
    Thus, $\mtx{Q}_k$ is the product of an upper Hessenberg matrix and an upper triangular matrix, and so it is also upper Hessenberg.
 
    Now, to show that $\mtx{A}^{(k+1)} = \mtx{R}_k \mtx{Q}_k + \mu_k \Id$ is upper Hessenberg, we need to prove that the entries below the first subdiagonal are zero.
    
    Consider entry $a_{i,j}^{(k+1)}$ of $\mtx{A}^{(k+1)}$, where $i > j + 1$:
    \begin{align*}
      \mtx{A}^{(k+1)} &= \mtx{R}_k \mtx{Q}_k + \mu_k \Id \\
      a_{i,j}^{(k+1)} &= (\mtx{R}_k \mtx{Q}_k)_{i,j} + (\mu_k \Id)_{i,j}\\
      &= (\mtx{R}_k \mtx{Q}_k)_{i,j}&\text{($\mu_k \Id$ diag. $\rightarrow (\mu_k \Id)_{i,j} = 0, \forall i \neq j$)}\\
      &= \sum_{p=1}^m r_{i,p}^{(k)} q_{p,j}^{(k)}
    \end{align*}
    Since $\mtx{R}_k$ is upper triangular, $r_{i,p}^{(k)} = 0$ for all $i > p$.
    Also, since $\mtx{Q}_k$ is upper Hessenberg, $q_{p,j}^{(k)} = 0$ for all $p > j + 1$.
    Combining these observations, we can conclude that $(\mtx{R}_k \mtx{Q}_k)_{i,j} = 0$ for all $i > j + 1$, and so $\mtx{A}^{(k+1)}$ is upper Hessenberg.

    Thus, we have shown that if $\mtx{A}^{(k)}$ is upper Hessenberg, then $\mtx{A}^{(k+1)}$ is also upper Hessenberg, which completes the induction.
 
  \subsection*{(b) [10 pts]}
  Prove that the total operation cost for each QR iteration is $O(m^2)$.

  \quad A single step of QR iteration involves a QR factorization and computing $\mtx{A}^{(k+1)}$ from the QR factors.

  \begin{enumerate}
    \item QR factorization:
    
    $\mtx{A}^{(k)} - \mu_k \Id$ is upper Hessenberg.
    We can compute its QR factorization using Givens rotations.
    Each Givens rotation zeros out one subdiagonal element in each column.
    Since there are $m-1$ nonzero subdiagonal elements, there will be $O(m-1)$ Givens rotations.
    Each Givens rotation requires $4m$ operations (2 multiplications and 2 additions per entry across two rows),
    giving a total cost of $4m(m-1) = O(m^2)$.
    \item Compute $\mtx{A}^{(k+1)} = \mtx{R}_k \mtx{Q}_k + \mu_k \Id$:
    
    $\mtx{R}_k \mtx{Q}_k$ is the product of an upper triangular matrix and an upper Hessenberg matrix, so we can compute the product using $O(m^2)$ operations.
    Adding the shift $\mu_k \Id$ takes $O(m)$ operations.
  \end{enumerate}

  Thus, the total operation cost for each QR iteration is $O(m^2) + O(m^2) + O(m) = O(m^2)$.

  \subsection*{(c) [10 pts]}
  In the QR step, we perform $m - 1$ Givens rotations on the matrix $\mtx{A}^{(k)} - \mu_k \Id$. 
  Suppose that after $m - 2$ Givens rotations, the bottom-right $2 \times 2$ sub-matrix of $\mtx{A}^{(k)} - \mu_k \Id$ is given by
  \begin{equation}
    \begin{pmatrix}
      a & b \\
      \varepsilon & 0
    \end{pmatrix}.
  \end{equation}
  Explain why the $(m, m)$ entry is $0$ at that stage, and prove that 
  \begin{equation}
    \mtx{A}_{m, m-1}^{(k + 1)} = - \frac{\varepsilon^2 b}{\varepsilon^2 + a^2}.
  \end{equation}

\quad Since we have applied $m-2$ Givens rotations, the $(m, m)$ entry is $0$ because each Givens rotation introduces a $0$ entry below the diagonal, and we have only one non-zero entry left in the last subdiagonal.

We are looking for the lower-left entry of the bottom-right $2\times2$ submatrix of the product $G^T(\mtx{A}^{(k)} - \mu_k \Id)G$, where $G$ is the Givens rotation matrix:
$$G \coloneqq \begin{pmatrix}
      c & -s \\
      s & c
\end{pmatrix},
$$
with $c \coloneqq \dfrac{a}{\sqrt{a^2 + \varepsilon^2}}$ and $s \coloneqq \dfrac{\varepsilon}{\sqrt{a^2 + \varepsilon^2}}$.

For brevity, let's denote the subscript $2\times2$ as the lower-right $2\times2$ submatrix of a matrix.

We have:
\begin{align*}
  \left(G^T(\mtx{A}^{(k)} - \mu_k \Id)G\right)_{2\times2} &= \left(G^T\mtx{A}^{(k)}G\right)_{2\times2} -\left(G^T \mu_k \Id G\right)_{2\times2}\\
  &=
  \begin{pmatrix}
      c & s \\
      -s & c
  \end{pmatrix}
  \begin{pmatrix}
      a & b \\
      \varepsilon & 0
  \end{pmatrix}
  \begin{pmatrix}
      c & -s \\
      s & c
  \end{pmatrix} +
  \begin{pmatrix}
    c & s \\
    -s & c
\end{pmatrix}
\begin{pmatrix}
    \mu_k & 0 \\
    0 & \mu_k
\end{pmatrix}
\begin{pmatrix}
    c & -s \\
    s & c
\end{pmatrix} 
\end{align*}

Note that we ultimately only care about the lower-left entry of the bottom-right $2\times2$ submatrix.

We can see the right term will contribute nothing to this entry:
$$
\begin{pmatrix}
  c & s \\
  -s & c
\end{pmatrix}
\begin{pmatrix}
  \mu_k & 0 \\
  0 & \mu_k
\end{pmatrix}
\begin{pmatrix}
  c & -s \\
  s & c
\end{pmatrix} = 
\begin{pmatrix}
  \cdots & \cdots \\
  \mu_k(-sc + cs) & \cdots 
\end{pmatrix} = 
\begin{pmatrix}
  \cdots & \cdots \\
  0 & \cdots 
\end{pmatrix}
$$

This leaves us with the left term to compute:
\begin{align*}
\begin{pmatrix}
  c & s \\
  -s & c
\end{pmatrix}
\begin{pmatrix}
  a & b \\
  \varepsilon & 0
\end{pmatrix}
\begin{pmatrix}
  c & -s \\
  s & c
\end{pmatrix}\\
\begin{pmatrix}
  ac + s\varepsilon & cb \\
  -sa + c\varepsilon & -sb
\end{pmatrix}
\begin{pmatrix}
  c & -s \\
  s & c
\end{pmatrix}\\
\begin{pmatrix}
  \cdots & \cdots \\
  (-sa + c\varepsilon)c + (-sb)s & \cdots
\end{pmatrix}
\end{align*}

Simplifying the lower-left entry and substituting for $c$ and $s$,
\begin{align*}
(-sa + c\varepsilon)c + (-sb)s &= -csa + c^2\varepsilon - s^2b\\
&= -\dfrac{a^2\varepsilon}{a^2 + \varepsilon^2} + \dfrac{a^2\varepsilon}{a^2 + \varepsilon^2} - \frac{\varepsilon^2 b}{\varepsilon^2 + a^2}\\
&= - \frac{\varepsilon^2 b}{\varepsilon^2 + a^2},
\end{align*}
which is our desired result.

  \subsection*{(d) [10 pts]} 
  Based on the previous result, explain why we can expect the single-shift QR algorithm to converge quadratically (provided that it is converging). 

  \quad In part (c), we derived that the off-diagonal entry of the bottom-right $2 \times 2$ sub-matrix of $\mtx{A}^{(k+1)}$ is given by
  $$-\frac{\varepsilon^2 b}{\varepsilon^2 + a^2}.$$
  The $\varepsilon^2$ is in the numerator means that as $\varepsilon \to 0$, the off-diagonal entry $\mtx{A}_{m, m-1}^{(k + 1)}$ will also approach $0$.
  That is, the off-diagonal entry decreases quadratically with respect to $\varepsilon$.
  Since the QR algorithm aims to drive the off-diagonal entries to $0$ to reveal the eigenvalues along the diagonal, this suggests the single-shift QR algorithm will converge quadratically if it converges at all.

  \subsection*{(e) [10 pts]}
  We showed that the single-shift QR algorithm converges quite fast if the guess is sufficiently accurate. 
  However, its convergence is not guaranteed.
  Give an example in which the single-shift QR algorithm fails to converge, and explain why.

\quad Consider the following matrix:

$$\mtx{A} = \begin{pmatrix}
  0 & 1 \\
  1 & 0
\end{pmatrix}$$

The true eigenvalues of $\mtx{A}$ are $\lambda_1 = 1$ and $\lambda_2 = -1$, and so $\left|\lambda_1\right| = \left|\lambda_2\right| = 1$.

Thus, the single-shift QR algorithm cannot decide which direction to move its estimate for the eigenvalues, and so it cannot proceed forward.

To see this, consider the first iteration of the single-shift QR algorithm applied to $\mtx{A}$.

\begin{itemize}
\item Initialize $\mtx{T}_0 = \mtx{A}$.
\item Since the diagonal entries are both zero, we choose $\mu_1 = 0$ as the shift (our "eigenvalue estimate").
\item Then, we take the QR factorization of $\mtx{T}_0 - \mu_1 \Id = \mtx{T}_0$.
\end{itemize}

We can already see that we can make no progress.
Furthermore, note that the off-diagonals are not zero (or near-zero), and never will be, and so we cannot use deflation to split up the problem and make progress that way either.

As explained in class, we need a way to break the symmetry of the problem, and this is why we introduce the Wilkinson Shift.

\section{Deflation upon Convergence [20 pts]}
Consider an upper Hessenberg matrix $\mtx{H} \in \R^{m \times m}$ with eigenvalue $\lambda$. 
We define 
\begin{align}
  \mtx{H} - \lambda \Id &= \mtx{U}_1 \mtx{R}_1 \quad \text{(QR factorization)} \\ 
  \mtx{H}_1 &= \mtx{R}_1 \mtx{U}_1 + \lambda \Id.
\end{align}

\subsection*{(a) [10 pts]}
Prove that if $\mtx{H}_{i + 1, i} \neq  0, \forall 1 \leq i < m$ ($\mtx{H}$ is an unreduced Hessenberg matrix), then 
\begin{equation}
  \mtx{H}_1\left(m, \colon\right) = \lambda e_m^{T}. 
\end{equation}

\begin{align*}
  \mtx{H} \vct{x} &= \lambda \vct{x}&\text{(eigenvector equation for $\mtx{H}$)}\\
  \mtx{U}_1^T (\mtx{H} - \lambda \Id) \vct{x} &= \mtx{U}_1^T \mtx{H} \vct{x} - \lambda \mtx{U}_1^T \vct{x} = \mtx{R}_1 \vct{y}&\text{($\vct{y} \coloneqq \mtx{U}_1^T \vct{x}$)}\\
  \mtx{H}_1 \vct{y} &= (\mtx{U}_1^T (\mtx{H} - \lambda \Id) \vct{x} + \lambda \Id) \vct{y}&\text{(substitute $\mtx{R}_1 \vct{y}$)}\\
  \mtx{H}_1 \vct{y} &= \mtx{U}_1^T \mtx{H} \vct{x} - \lambda \mtx{U}_1^T \vct{x} + \lambda \vct{y}&\text{(simplify)}\\
  \mtx{H}_1 \vct{y} &= \mtx{U}_1^T \lambda \vct{x} - \lambda \mtx{U}_1^T \vct{x} + \lambda \vct{y}&\text{(substitute $\mtx{H} \vct{x} = \lambda \vct{x}$)}\\
  \mtx{H}_1 \vct{y} &= \lambda \vct{y}&\text{(simplify)}\\
  h_{m,m-1} y_{m-1} + h_{m,m} y_m &= \lambda y_m&\text{(examine the last element)}
\end{align*}

We can now conclude that $h_{m,m} = \lambda$ and $h_{m,m-1} = 0$, and so
$$\mtx{H}_1\left(m, \colon\right) = \begin{bmatrix} 0 & \cdots & 0 & 0 & \lambda \end{bmatrix} = \lambda e_m^{T}.$$

\subsection*{(b) [10 pts]}
Explain the connection between this result and the process of deflation in the QR iteration algorithm. 

\quad When an off-diagonal entry of the matrix in the QR iteration converges to near-zero, we can "deflate" the matrix by dividing it into two smaller matrices, and applying the QR iteration to each smaller matrix separately.

In problem (a), we found that if $\mtx{H}_{m, m-1}$ converges to zero, the last row of the matrix $\mtx{H}_1$ becomes $\lambda e_m^T$.
This means that the matrix $\mtx{H}_1$ can be partitioned into two smaller matrices, one of size $(m-1) \times (m-1)$ and the other of size $1 \times 1$.
The $1 \times 1$ matrix contains the converged eigenvalue $\lambda$ and is deflated from the rest of the matrix.
The QR iteration can then be applied to the smaller $(m-1) \times (m-1)$ matrix.

\section{An implicit QR Factorization [15 bonus pts]}
Denote $\mtx{H} = \mtx{H}_1$, and assume we generate a sequence of matrices $\mtx{H}_k$ via 
\begin{equation}
  \mtx{H}_k - \mu_k \Id = \mtx{U}_k \mtx{R}_k, \quad \mtx{H}_{k + 1} = \mtx{R}_k \mtx{U}_k + \mu_k \Id. 
\end{equation}
Prove that 
\begin{equation}
  \left(\mtx{U}_1 \cdots \mtx{U}_j\right)\left(\mtx{R}_j \cdots \mtx{R}_1\right) = \left(\mtx{H} - \mu_{j} \Id\right) \cdots \left(\mtx{H} - \mu_1 \Id\right). 
\end{equation}
This result shows that we are implicitly computing a QR factorization of 
\begin{equation}
  \left(\mtx{H} - \mu_j \Id\right) \cdots \left(\mtx{H} - \mu_1 \Id\right).   
\end{equation}

\quad We will proove this result by induction on $k$.

\textbf{Base case ($k = 1$):}
\begin{align*}
  \mtx{U}_k \mtx{R}_k &= \mtx{H}_k - \mu_k \Id&\text{(given)}\\
  \mtx{U}_1 \mtx{R}_1 &= \mtx{H}_1 - \mu_1 \Id&\text{(set $k = 1$)}\\
  \left(\mtx{U}_1\right)\left(\mtx{R}_1\right) &= \left(\mtx{H} - \mu_1 \Id\right)&\text{(since $\mtx{H} \coloneqq \mtx{H}_1$. QED for base case)}
\end{align*}

\textbf{Inductive step:}
Assume that the result holds for $k > 0$.

Define $\mtx{Q}_k \coloneqq \left(\mtx{U}_1 \cdots \mtx{U}_k\right)$, $\mtx{P}_k \coloneqq \left(\mtx{R}_k \cdots \mtx{R}_1\right)$, and $\mtx{G}_k \coloneqq \left(\mtx{H} - \mu_k \Id\right) \cdots \left(\mtx{H} - \mu_1 \Id\right)$.

Then, our inductive assumption is
\begin{align*}
  \left(\mtx{U}_1 \cdots \mtx{U}_k\right)\left(\mtx{R}_k \cdots \mtx{R}_1\right) &= \left(\mtx{H} - \mu_k \Id\right) \cdots \left(\mtx{H} - \mu_1 \Id\right)\\
  \mtx{Q}_k\mtx{P}_k &= \mtx{G}_k.
\end{align*}
We want to show that the result also holds for $k+1$.
That is, we want to show that
\begin{align*}
  \left(\mtx{U}_1 \cdots \mtx{U}_{k+1}\right)\left(\mtx{R}_{k+1} \cdots \mtx{R}_1\right) &= \left(\mtx{H} - \mu_{k+1} \Id\right) \cdots \left(\mtx{H} - \mu_1 \Id\right)\\
  \left(\mtx{U}_1 \cdots \mtx{U}_{k} \mtx{U}_{k+1}\right)\left(\mtx{R}_{k+1} \mtx{R}_{k} \cdots \mtx{R}_1\right) &=  \left(\mtx{H} - \mu_{k+1} \Id\right) \left(\mtx{H} - \mu_k \Id\right) \cdots \left(\mtx{H} - \mu_1 \Id\right)\\
  \mtx{Q}_k \mtx{U}_{k+1}\mtx{R}_{k+1} \mtx{P}_k &= \left(\mtx{H} - \mu_{k+1} \Id\right) \mtx{G}_k.
\end{align*}

First, observe the following relation:
\begin{align*}
  \mtx{H}_k - \mu_k \Id &= \mtx{U}_k \mtx{R}_k&\text{(given)}\\
  \left(\mtx{U}_k\right)^T \mtx{H}_k \mtx{U}_k &=\left(\mtx{U}_k\right)^T \left(\mtx{U}_k \mtx{R}_k + \mu_k \Id\right) \mtx{U}_k&\text{(mult. left and right by $\left(\mtx{U}_k\right)^T$ and $\mtx{U}_k$)}\\
  &= \mtx{R}_k \mtx{U}_k + \mu_k \Id&\text{(since $\mtx{U}_k$ is orthonormal)}\\
  \left(\mtx{U}_k\right)^T \mtx{H}_k \mtx{U}_k &= \mtx{H}_{k+1}&\text{(by definition of $\mtx{H}_{k+1}$)}
\end{align*}

\textit{I am not sure how to progress from here. Looking forward to reading the answer!}

\section{QR with Shifts [30 pts]}
\subsection*{(a) Almost upper triangular [7.5 pts]} 
Go to section (a) of the file \texttt{HW6\_your\_code.jl} and implement a function that reduces a symmetric matrix $\mtx{A} \in \mathbb{R}^{m\times m}$ to Hessenberg form using Householder reflections. You should end up with a matrix $\mtx{T}$ in Hessenberg form. Your algorithm should operate in place, overwriting the input matrix and not allocating additional memory. 

\subsection*{(b) Givens [7.5 pts]} 
Go to section (b) of the file \texttt{HW6\_your\_code.jl} and implement a function that runs a single iteration of the unshifted QR algorithm. Your function should take $\mtx{T}_{k}$ in Hessenberg form as an input and compute $\mtx{T}_{k+1}$ also in Hessenberg form. You should use Givens rotations to implement QR-factorization on $\mtx{T}_{k}$.

\subsection*{(c) Single-Shift vs. Wilkson Shifts [7.5 pts]}
Go to section (c) of the file \texttt{HW6\_your\_code.jl} and implement a function that runs the practical QR iteration with both the Single-Shift and Wilkinson Shift. Your function should have an input that allows you to select which type of shift you want to use.  Your implementation should include deflation and a reasonable criteria for when to implement deflation and terminate your QR iterations. You can use your function from part (b) to do the QR iteration at each step.

\subsection*{(d) Breaking symmetry [7.5 pts]}
Go to section (d) of the file \texttt{HW6\_your\_driver.jl} and design an experiment that evaluates your practical QR algorithm with shifts. You should include a semi-log plot showing the rate of convergence of your algorithm using Single-Shift and Wilkinson Shift. Compare the results with the rate of convergence you expected to see for both cases. Do you have a preference between the Wilkinson shift or the Rayleigh shift?  If so which one do you prefer and why?

\quad \textit{I was not able to complete this.} I struggled to ensure that the two shifts both resulted in the same selected eigenvalue to converge for comparison, and I wasn't able to produce anything meaningful in the time I had.

\end{document}
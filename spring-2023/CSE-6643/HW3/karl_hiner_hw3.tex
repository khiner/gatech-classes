\documentclass[twoside,10pt]{article}
%=================================================
% Basics
%=================================================
\usepackage{fixltx2e} % Makes \( \) equation style robust, among other
                      % things. Must be the first package.


% Makes ligatured fonts searchable and copyable in pdf readers
\usepackage{cmap} % Load before fontenc 

% Always include these font encodings in your document 
% unless you have a very good reason.
\usepackage[T1]{fontenc}
\usepackage[utf8]{inputenc}

\usepackage{verbatim}

%=============
% Fonts
%=============

\usepackage{lmodern} % Improved version of computer modern
\usepackage[scale=0.88]{tgheros} % Helvetica clone for sans serif font


\newcommand\hmmax{2} % Default is 3.
\newcommand\bmmax{2} % Default is 4.

\usepackage{bm} % boldmath must be called after the package
\providecommand{\mathbold}[1]{\bm{#1}}

%=============
% AMS Packages and fonts
%=============
\usepackage{amsmath,amsbsy,amsgen,amscd,amsthm,amsfonts,amssymb} 

%=============
% Margins and paper size
%=============
\usepackage[centering,top=1.5in,bottom=1.2in,left=1.4in,right=1.4in]{geometry}
\usepackage{parskip}


%=============
% Section headings
%=============
\usepackage[sf,bf,compact]{titlesec}

%=============
% Tables and lists
%=============
\usepackage{booktabs,longtable,tabu} % Nice tables
\setlength{\tabulinesep}{1mm}
\usepackage[font=small,margin=10pt,labelfont={sf,bf},labelsep={space}]{caption}

%=============
% Code output
%=============
% \usepackage{listings}
% \usepackage{minted}




\usepackage{enumitem}
\setitemize{itemsep=0pt} 
\setenumerate{itemsep=0pt}
\setlist{labelindent=\parindent,%  % Recommended by enumitem package
  font=\sffamily}


%=============
% Hyperlink colors
%=============
\usepackage[usenames,dvipsnames]{xcolor}
\definecolor{steelblue}{HTML}{A1BDC7}
\definecolor{orange}{HTML}{D98C21}
\definecolor{silver}{HTML}{B0ABA8}
\definecolor{rust}{HTML}{B8420F}
\definecolor{seagreen}{HTML}{2E6B69}
\definecolor{joshua}{HTML}{FBDC7F}
\definecolor{darksky}{HTML}{154c79}

\colorlet{steelblue}{silver!30!white}
\colorlet{darkorange}{orange!85!black}
\colorlet{darksilver}{silver!85!black}
\colorlet{darksteelblue}{steelblue!85!black}
\colorlet{darkrust}{rust!85!black}
\colorlet{darkseagreen}{seagreen!85!black}

\usepackage{url}
\usepackage[colorlinks=true]{hyperref}
\hypersetup{linkcolor=darkrust}    
\hypersetup{citecolor=darkseagreen}      
\hypersetup{urlcolor=darksilver}     

%=============
% Microtype
%=============
\usepackage[final]{microtype} 

%=====================
% Header
%=====================
% \usepackage{fancyhdr}
% \usepackage{nopageno} % Gets rid of page number at the bottom
% \fancyhf{} % Clear header style
% \renewcommand{\headrulewidth}{0.5pt} % remove the header rule
% \pagestyle{fancy}
% \fancyhead[LE,RO]{\textsf{\small \thepage}}
% 
% \setlength{\headheight}{14pt}
%=====================
% Fix delimiters
%=====================

% Fixes \left and \right spacing issues. See discussion at
% http://tex.stackexchange.com/questions/2607/spacing-around-left-and-right
\let\originalleft\left
\let\originalright\right
\renewcommand{\left}{\mathopen{}\mathclose\bgroup\originalleft}
\renewcommand{\right}{\aftergroup\egroup\originalright}

%=================================================
% Math macros
%=================================================

%=============
% Generalities
%=============
\usepackage{mathtools}
\mathtoolsset{centercolon}  % Makes := typeset correctly for definitions

%%% Equation numbering
%\numberwithin{equation}{section} 

%%% Annotations
\newcommand{\notate}[1]{\textcolor{red}{\textbf{[#1]}}}

%==============
% Symbols
%==============
\let\oldphi\phi
\let\oldeps\epsilon

\renewcommand{\phi}{\varphi}
\renewcommand{\epsilon}{\varepsilon}
\newcommand{\eps}{\varepsilon}

%==============
% Constants
%==============

% Set constants upright
\newcommand{\cnst}[1]{\mathrm{#1}}  
\newcommand{\econst}{\mathrm{e}}

\newcommand{\zerovct}{\vct{0}} % Zero vector
\newcommand{\Id}{\mathbf{I}} % Identity matrix
\newcommand{\onemtx}{\bm{1}}
\newcommand{\zeromtx}{\bm{0}}

%==============
% Sets
%==============
\providecommand{\mathbbm}{\mathbb} % In case we don't load bbm

% Reals, complex, naturals
\newcommand{\R}{\mathbbm{R}}
\newcommand{\C}{\mathbbm{C}}
\newcommand{\K}{\mathbbm{K}}
\newcommand{\N}{\mathbbm{N}}

%==============
% Probability
%==============
\newcommand{\Prob}{\operatorname{\mathbbm{P}}}
\newcommand{\Expect}{\operatorname{\mathbb{E}}}

%==============
% Vectors and matrices 
%==============
\newcommand{\vct}[1]{\mathbold{#1}}
\newcommand{\mtx}[1]{\mathbold{#1}}

\newcommand{\mrange}{\operatorname{range}}
\newcommand{\mnull}{\operatorname{null}}


\DeclareMathOperator*{\argmax}{argmax}
\DeclareMathOperator*{\argmin}{argmin}

\bibliographystyle{plain}

\begin{document}
\nocite{*}

\title{CSE 6643 Homework 3}
\author{Karl Hiner}
\date{}
\maketitle

\section{One-upping [25 pts]}
Let $\mtx{A} \in \R^{m \times m}$ have full rank. Assume that we have already computed the QR decomposition of $\mtx{A}$.
For $\mtx{u}, \mtx{v} \in \R^m$, we call the matrix $\mtx{B} = \mtx{A} + \mtx{u}\mtx{v}^{T}$ a rank-1 update of $\mtx{A}$. 

\subsection*{(a) [5 pts]}
Prove that if $\vct{v}^{T} \mtx{A}^{-1} \vct{u} \neq - 1$, then $\mtx{B}$ is invertible.

\quad Let $\vct{x} \in \R^{m}$ be a non-zero vector. Then,
\begin{align*}
  \vct{v}^{T} \mtx{A}^{-1} \vct{u} &\neq -1\\
  1 &\neq -\vct{v}^{T} \mtx{A}^{-1} \vct{u} \\
  \vct{v}^{T} \vct{x} &\neq -(\vct{v}^{T} \mtx{A}^{-1} \vct{u})(\vct{v}^{T} \vct{x}) \\
  \vct{x} &\neq -\mtx{A}^{-1} \mtx{u}(\vct{v}^{T} \vct{x}) \\
  \mtx{A} \vct{x} &\neq -\mtx{u}\vct{v}^{T} \vct{x} \\
  \left(\mtx{A} + \mtx{u}\vct{v}^{T}\right)\vct{x} &\neq \mtx{0}. \\
  \mtx{B}\vct{x} &\neq \vct{0}
\end{align*}

\quad Thus, if $\vct{v}^{T} \mtx{A}^{-1} \vct{u} \neq -1$, there are no nontrivial solutions for $\mtx{B}\vct{x} = \vct{0}$, and so $\mtx{B}$ is invertible.

\subsection*{(b) [10 pts]} 
Design an algorithm that provably solves the system of equations $\mtx{B} \mtx{x} = \mtx{b}$ in $O(m^2)$ operations. 
\begin{align*}
  \vct{x} &= \mtx{B}^{-1}\vct{b}&\text{(assume $\mtx{B}$ is invertible)}\\
  &= \left(\mtx{A} + \vct{u}\vct{v}^T\right)^{-1}\vct{b}&\text{(since $\mtx{B} = \mtx{A} + \vct{u}\vct{v}^T$)}\\
  &= \left(\mtx{A}^{-1} - \dfrac{\mtx{A}^{-1}\vct{u}\vct{v}^T\mtx{A}^{-1}}{1+\vct{v}^T\mtx{A}^{-1}\vct{u}}\right)\vct{b}&\text{(Sherman–Morrison formula)}\\
  &= \mtx{A}^{-1}\vct{b} - \dfrac{\mtx{A}^{-1}\vct{u}\vct{v}^T}{1+\vct{v}^T\mtx{A}^{-1}\vct{u}}\mtx{A}^{-1}\vct{b}&\text{(note: (1a above), $\mtx{B}$ invertible $\implies$ den $\neq 0$)}\\
  &= \tilde{\vct{x}} - \dfrac{\tilde{\vct{u}}\vct{v}^T}{1+\vct{v}^T\tilde{\vct{u}}}\tilde{\vct{x}}&\text{(let $\tilde{\vct{x}} \equiv \mtx{A}^{-1}\vct{b}, \tilde{\vct{u}} \equiv \mtx{A}^{-1}\vct{u}$)}\\
  &= \left[\vct{I} + \left(\dfrac{-1}{1+\vct{v}^T\tilde{\vct{u}}}\right)\vct{\tilde{\vct{u}}\vct{v}^T}\right]\tilde{\vct{x}}&\text{(rearrange)}\\
  &= \left(\vct{I} + \alpha\tilde{\vct{u}}\vct{v}^T\right)\tilde{\vct{x}}&\text{(let $\alpha \equiv \dfrac{-1}{1+\vct{v}^T\tilde{\vct{u}}}$)}
\end{align*}

\quad Thus, we have derived an expression for $\vct{x}$ in terms of a scalar $\alpha$ and vectors $\tilde{\vct{u}}, \tilde{\vct{x}}$.

\quad Vectors $\tilde{\vct{u}}$ and $\tilde{\vct{x}}$ are both defined in terms of $\mtx{A}^{-1} = \mtx{R}^{-1}\mtx{Q}^T$, where we assume $\mtx{R}$ and $\mtx{Q}$ have already been computed.
Thus, both vectors can be computed using back substitution in $O(m^2)$.
$\alpha$ can then be computed in $O(m)$, and the final expression for $\vct{x}$ involves a vector-scalar product, a vector outer product, an identity addition, and a matrix-vector multiplication, for a total of
$$O(m) + O(m^2) + O(m) + O(m^2) = O(2m) + O(m^2) = O(m^2)$$
operations.

\quad Here is the algorithm:
\begin{enumerate}
  \item Solve $\mtx{R}\tilde{\vct{x}} = \mtx{Q}^T\vct{b}$ for $\tilde{\vct{x}}$ using back substitution. ($O(m^2)$)
  \item Solve $\mtx{R}\tilde{\vct{u}} = \mtx{Q}^T\vct{u}$ for $\tilde{\vct{u}}$ using back substitution. ($O(m^2)$)
  \item Compute the scalar $\alpha = \dfrac{-1}{1+\vct{v}^T\tilde{\vct{u}}}.$ ($O(m)$)
  \item Finally, compute the solution $\vct{x} = \left(\vct{I} + \left(\left(\alpha\tilde{\vct{u}}\right)\vct{v}^T\right)\right)\tilde{\vct{x}}.$ ($O(m^2)$)
\end{enumerate}

\quad Since the highest-order term across all steps is $O(m^2)$, the total number of operations is thus $O(m^2)$.

\subsection*{(c) r-upping [10 pts]}
Extend the algorithm from the previous exercise to the case of $\mtx{B} = \mtx{A} + \mtx{U}\mtx{V}^{T}$, for $\mtx{U},\mtx{V} \in \R^{m \times r}$ and $r \ll m$.
Calculate the asymptotic complexity of the resulting algorithm.

\quad The Sherman-Morrison formula was the key step above.
Here, we will use the generalization of that formula, the Woodbury matrix identity.
Given the definitions above, the Woodbury matrix identity states that if $\left(\mtx{I} + \mtx{V}^T\mtx{A}^{-1} \mtx{U}\right)$ is invertible,
$$\mtx{B}^{-1} = \mtx{A}^{-1} - \mtx{A}^{-1}\mtx{U} \left(\mtx{I} + \mtx{V}^T\mtx{A}^{-1}\mtx{U}\right)^{-1} \mtx{V}^T\mtx{A}^{-1}.$$

Thus, we can express the solution $\vct{x}$ as
\begin{align*}
  \vct{x} &= \left(\mtx{A}^{-1} - \mtx{A}^{-1}\mtx{U} \left(\mtx{I} + \mtx{V}^T\mtx{A}^{-1}\mtx{U}\right)^{-1} \mtx{V}^T\mtx{A}^{-1}\right)\vct{b}\\
  &= \mtx{A}^{-1}\vct{b} - \mtx{A}^{-1}\mtx{U} \left(\mtx{I} + \mtx{V}^T\mtx{A}^{-1}\mtx{U}\right)^{-1} \mtx{V}^T\mtx{A}^{-1}\vct{b}\\
  &= \tilde{\vct{x}} - \tilde{\mtx{U}} \left(\mtx{I} + \mtx{V}^T\tilde{\mtx{U}}\right)^{-1} \mtx{V}^T\tilde{\vct{x}} &\text{(let $\tilde{\vct{x}} \equiv \mtx{A}^{-1}\vct{b}, \tilde{\mtx{U}} \equiv \mtx{A}^{-1}\mtx{U}$)}\\
  &= \left[\mtx{I} - \tilde{\mtx{U}} \left(\mtx{I} + \mtx{V}^T\tilde{\mtx{U}}\right)^{-1} \mtx{V}^T\right]\tilde{\vct{x}} &\text{(rearrange)}\\
  &= \left[\mtx{I} - \left(\mtx{I} + \tilde{\mtx{U}}\mtx{V}^T\right)^{-1}\tilde{\mtx{U}}\mtx{V}^T\right]\tilde{\vct{x}} &\text{("push-through identity"\cite{henderson_deriving_1981})}\\
  &= \left(\mtx{I} + \mtx{\alpha}\tilde{\mtx{U}}\mtx{V}^T\right)\tilde{\vct{x}} &\text{(let $\mtx{\alpha} =- \left(\mtx{I} + \tilde{\mtx{U}}\mtx{V}^T\right)^{-1}$)}
\end{align*}

Note that, since $\tilde{\mtx{U}} \in \R^{m \times r}$, and $\mtx{V}^T \in \R^{r \times m}$, we can compute the matrix product $\tilde{\mtx{U}}\mtx{V}^T$ using $O(m^2r)$ operations.

TODO need to finish after step 2.
https://people.clas.ufl.edu/hager/files/update-1.pdf p224 is close.

\quad Here is the algorithm:
\begin{enumerate}
  \item Solve $\mtx{R}\tilde{\vct{x}} = \mtx{Q}^T\vct{b}$ for $\tilde{\vct{x}}$ using back substitution (same as step 1 in (b)). ($O(m^2)$)
  \item Compute $\tilde{\mtx{U}} = \mtx{A}^{-1}\mtx{U}$ by solving each of the linear systems $\mtx{R}\tilde{\vct{u}}_i = \mtx{Q}^T\vct{u}_i$ for $\tilde{\vct{u}}_i$, where $\tilde{\vct{u}}_i$ is the $i$th column of $\tilde{\mtx{U}}$ and $\vct{u}_i$ is the $i$th column of $\mtx{U}$.
  This can be done using back substitution (as in step 2 in (b)) for each of the $r$ columns. ($O(m^2r)$)
  \item Compute the scalar $\alpha = \dfrac{-1}{1+\vct{v}^T\tilde{\vct{u}}}.$ ($O(m)$)
  \item Finally, compute the solution $\vct{x} = \left(\vct{I} + \left(\left(\alpha\tilde{\vct{u}}\right)\vct{v}^T\right)\right)\tilde{\vct{x}}.$ ($O(m^2)$)
\end{enumerate}

\section{You Factor [25 pts]}
In class we have seen that if $\tilde{\vct{x}} \in \R^{m}$ is the solution to the system $\mtx{A}\vct{x} = \vct{b}$, as computed by unpivoted LU factorization, we have  
\begin{equation}
  \left(\mtx{A} + \mtx{E}\right)\tilde{\vct{x}} = \vct{b}.  
\end{equation}
For $u$ the unit roundoff error and $\tilde{\mtx{L}}, \tilde{\mtx{U}}$ the LU factors computed in finite precision, we have
\begin{equation}
  \left|\mtx{E}\right| \leq m u \left(2 \left|\mtx{A}\right| + 4 \left|\tilde{\mtx{L}} \right| \left|\tilde{\mtx{U}}\right|\right) + O(u^2).
\end{equation}
Here, $|\cdot|$ signifies the element-wise absolute values and $\leq$ is interpreted element-wise, as well.
In this problem, we investigate the conclusions from this bound in the case of row-pivoted LU factorization.

Generally useful things:
\begin{itemize}
  \item p174: Because each pivot selection involves maximization over a column, this algorithm produces a matrix $L$ with entries of absolute value $\leq 1$ everywhere below the diagonal.
      This implies $\|L\| = O(1)$ in any norm.
      ... so the algorithm is backward stable if $\|U\| = O(\|A\|)$.
\end{itemize}

\subsection*{(a) [7.5 pts]}
Deduce that under row-pivoted $LU$ factorization and taking $\| \cdot \|_{\infty}$ to signify the vector-infinity norm, we have 
\begin{equation}
  \left\|\mtx{E}\right\|_{\infty} \leq m u \left(2 \left\|\mtx{A}\right\|_{\infty} + 4 m \left\|\tilde{\mtx{U}}\right\|_{\infty} \right) + O(u^2).
\end{equation}
This prompts us to investigate the growth factor $\rho \coloneqq \frac{\left\| \mtx{U} \right\|_{\infty}}{\left\|\mtx{A}\right\|_{\infty}}$ of row-pivoted LU factorization.

\subsection*{(b) [5 pts]}
Verify that the rows $\vct{u}_{i}^{T}, \vct{a}_{i}^{T}$ of $\mtx{U}, \mtx{A}$ satisfy 
\begin{equation}
  \vct{u}_{i}^{T} = \vct{a}_i^T - \sum \limits_{j = 1}^{i - 1} \mtx{L}_{ij} \vct{u}_{j}^T.
\end{equation}

Given the definition of row-pivoted LU factorization, we can write the system of linear equations as 
$\mtx{A} \vct{x} = \vct{b} \implies \mtx{P} \mtx{A} \vct{x} = \mtx{P} \vct{b} \implies \mtx{L} \mtx{U} \vct{x} = \vct{b}$
Since $\mtx{L}$ is a lower triangular matrix, $\mtx{U}$ is an upper triangular matrix, with the diagonal elements equal to 1.
So, we can write the i-th row of $\mtx{U}$ as 
$\vct{u}_{i}^{T} = \vct{a}_i^T - \sum \limits_{j = 1}^{i - 1} \mtx{L}_{ij} \vct{u}_{j}^T$
which satisfies the equation (3).

\subsection*{(c) [5 pts]}
Use part (b) to show that $\left\|\tilde{\mtx{U}}\right\|_{\infty} \leq 2^{m - 1} \left\| \mtx{A}\right\|_{\infty}$. 


This is the same as asking to prove that $\max{\rho} = 2^{m-1}$ for an $mxm$ matrix $\mtx{A}$ (Exercise 22.1 in the book).
Note that (d) below is an example of this worst case.
Useful here: p174: Because each pivot selection involves maximization over a column, this algorithm produces a matrix $L$ with entries of absolute value $\leq 1$ everywhere below the diagonal.
Probably want in our proof that there are no pivots in this worst case (as in the example below).

From canvas: We can first bound each element of $u_1^T$ using (b), and then bound each element of $u_2^T$ using (b) and the bound for each element of $u_1^T$, and then bound each element of $u_3^T$ using (b) and the bounds for $u_1^T, u_2^T$, and so on...

\subsection*{(d) [7.5 pts]} Consider matrices of the form 
\begin{equation}
  \mtx{A} = 
  \begin{pmatrix}
    1  &    &   &       & 1 \\
    -1 & 1  &   &       & 1 \\
    -1 & -1 & 1 &       & 1 \\
    -1 & -1 & -1 &    1 & 1 \\
    -1 & -1 & -1 &  - 1 & 1 
  \end{pmatrix}.
\end{equation}
Derive the growth factor in this case as a function of $m$.

The $\mtx{P}\mtx{A} = \mtx{L}\mtx{U}$ factorization for $\mtx{A}$ is:
\begin{equation}
  \begin{pmatrix}
    1  &    &   &       & 1 \\
    -1 & 1  &   &       & 1 \\
    -1 & -1 & 1 &       & 1 \\
    -1 & -1 & -1 &    1 & 1 \\
    -1 & -1 & -1 &  - 1 & 1 
  \end{pmatrix}
  =
  \begin{pmatrix}
    1  &    &   &        \\
    -1 & 1  &   &        \\
    -1 & -1 & 1 &        \\
    -1 & -1 & -1 &    1  \\
    -1 & -1 & -1 &  - 1 & 1 
  \end{pmatrix}
  \begin{pmatrix}
    1  &   &   &   & 1 \\
       & 1 &   &   & 2 \\
       &   & 1 &   & 4 \\
       &   &   & 1 & 8 \\
       &   &   &   & 16 
  \end{pmatrix}
\end{equation}

The growth factor is then

$$\rho = \dfrac{\left\| \mtx{U} \right\|_{\infty}}{\left\|\mtx{A}\right\|_{\infty}} = \dfrac{16}{1} = 16 = 2^4 = 2^{m - 1}.$$

How does this relate to part (c)? 

\section{[25 pts]}
Suppose that $\mtx{A} \in \K^{m \times m}$ is strictly column diagonally dominant, meaning that for all $1 \leq k \leq m$, 
\begin{equation}
  \left|\mtx{A}_{kk}\right| > \sum \limits_{j \neq k} \left|\mtx{A}_{jk}\right|.
\end{equation}
Show that if LU factorization with row pivoting is applied to $\mtx{A}$, no row interchange takes place.

To show that now row interchange takes place, we can show the following:
\begin{enumerate}
  \item Show that $\mtx{A}$'s strict column diagonal property implies no row interchange takes place during the first factorization $k = 1$.
    Then the first permutation matrix $\mtx{P}^1 = \mtx{I}$, and we have $\mtx{L}^1\mtx{P}^1\mtx{A} = \mtx{L}^1\mtx{A} = \mtx{U}^1$, with $\mtx{U}^1$ being the first intermediate matrix on the way to $\mtx{U}$ after completing factorization step $k=1$.
    Express $\mtx{U}^1$ as
    $$\mtx{U}^1 = \begin{bmatrix}\mtx{A}_{1 1} & \mtx{A}_{1 j>1}\\\hbox{}\\\mtx{0} & \mtx{A}^{1}\end{bmatrix},$$
    with $\mtx{A}^1 = \mtx{A}_{2:m,2:m} \in \K^{m-1 \times m-1}$.
  \item Given (1), show that $\mtx{A}^{1}$ is also strictly diagonally dominant.
    Then, apply LU facorization with row pivoting to $\mtx{A}^1$, with $\hat{\mtx{P}}\mtx{A}^1 = \hat{\mtx{L}}\hat{\mtx{U}}$.
    We conclude from (1) that $\hat{\mtx{P}}^1 = \mtx{I}$, and we have $\hat{\mtx{L}}^1\mtx{A}^1 = \hat{\mtx{U}}^1$, where
    $$\hat{\mtx{U}}^1 = \begin{bmatrix}\mtx{A}^1_{1 1} & \mtx{A}^1_{1,j>1}\\\hbox{}\\\mtx{0} & \mtx{A}^{2}\end{bmatrix},$$
    with $\mtx{A}^2 = \mtx{A}^1_{2:,2:} \in \K^{m-2 \times m-2}$.
\end{enumerate}

If we show (1) and (2) hold, then we can apply step (1) again to matrix $\mtx{A}^2$ to show that no row interchange takes place at step $k = 2$.
Then we can apply step again (2) to show $\mtx{A}^3$ is also strictly diagonally dominant, etc.
Thus, if we show (1) and (2), we can inductively infer that no row interchanges take place during all factorization steps $1 \leq k \leq m - 1$.

\begin{enumerate}
  \item Show that $\left|\mtx{A}_{kk}\right| > \sum \limits_{j \neq k} \left|\mtx{A}_{jk}\right| \implies \mtx{P}^1 = \mtx{I}$:
  
  During the first factorization step, we choose a pivot row $j_1 = \argmax\limits_{j}{\left|\mtx{A}_{j 1}\right|}$.
  \begin{align*}
    \left|\mtx{A}_{kk}\right| &> \sum \limits_{j \neq k} \left|\mtx{A}_{jk}\right| > \max\limits_{j \neq k} \left|\mtx{A}_{jk}\right| \implies\\
    \left|\mtx{A}_{kk}\right| &= \max\limits_{j} \left|\mtx{A}_{jk}\right| \implies\\
    \left|\mtx{A}_{1 1}\right| &= \max\limits_{j} \left|\mtx{A}_{j 1}\right| \implies\\
    1 &= \argmax\limits_{j}{\left|\mtx{A}_{j 1}\right|},
  \end{align*}
  and so we choose the first pivot row $j_1 = 1$.
  In other words, no rows are swapped during this first step, and $\mtx{P}^1 = \mtx{I}$.
  \item Show that $\mtx{A}^{1}$ is also strictly diagonally dominant.

  Let $\alpha \equiv \dfrac{\mtx{A}_{1 j}}{\mtx{A}_{1 1}}$.
  Here is what the $j$th column of $\mtx{A}^1$ looks like:
  $$\begin{bmatrix*}
    \mtx{A}_{2 j}-\alpha\mtx{A}_{2 1}\\
    \cdots\\
    \mtx{A}_{j j}-\alpha\mtx{A}_{j 1}\\
    \cdots\\
    \mtx{A}_{m j}-\alpha\mtx{A}_{m 1}
  \end{bmatrix*}$$
  Thus, for $\mtx{A}^1$ to be strictly diagonally dominant, we must have
  \begin{align*}
    \left|\mtx{A}^1_{kk}\right| &> \sum \limits_{j \neq k} \left|\mtx{A}^1_{jk}\right| \implies\\
    \left|\mtx{A}_{j j}-\alpha\mtx{A}_{j 1}\right| &> \sum_{i \geq 2, i \neq j}\left|\mtx{A}_{i j}-\alpha\mtx{A}_{i 1}\right|.
  \end{align*}
  Starting with the LHS:
  \begin{align*}
    \left|\mtx{A}_{j j}-\alpha\mtx{A}_{j 1}\right| &\geq \left|\mtx{A}_{j j}\right|-\left|\alpha\mtx{A}_{j 1}\right|\\
    &> \sum_{i \neq j}\left|\mtx{A}_{i j}\right|-\left|\alpha\mtx{A}_{j 1}\right|&\text{($\mtx{A}$ is SDD)}\\
    &= \sum_{i \geq 2, i \neq j}\left|\mtx{A}_{i j}\right|-\left|\mtx{A}_{1 j}\right| -\left|\alpha\mtx{A}_{j 1}\right|&\text{(move $i = 1$ out of sum)}\\
    &= \sum_{i \geq 2, i \neq j}\left|\mtx{A}_{i j}\right|-\left|\alpha\mtx{A}_{1 1}\right| -\left|\alpha\mtx{A}_{j 1}\right|&\text{(using def. of $\alpha$)}\\
    &> \sum_{i \geq 2, i \neq j}\left|\mtx{A}_{i j}\right|-\sum_{i \neq 1}\left|\alpha\mtx{A}_{i 1}\right| -\left|\alpha\mtx{A}_{j 1}\right|&\text{($\mtx{A}$ is SDD)}\\
    &= \sum_{i \geq 2, i \neq j}\left|\mtx{A}_{i j}\right|-\sum_{i \geq 2, i \neq j}\left|\alpha\mtx{A}_{i 1}\right|&\text{(combine right two terms)}\\
    &= \sum_{i \geq 2, i \neq j}\left|\mtx{A}_{i j}-\alpha\mtx{A}_{i 1}\right|&\text{(QED)}
  \end{align*}
  \begin{align*}
  \end{align*}
\end{enumerate}

\section{Pivoting [25 pts]}
\subsection*{(a) [5 pts]} 
Go to section (a) of the file \texttt{HW3\_your\_code.jl} and implement a function that takes in a matrix $\mathtt{LU} \in \K^{m \times m}$ containing the upper triangular part of $\mtx{U}$ as well as the strict lower triangular part of $\mtx{L}$, as well as an array $\mathtt{P} \in \{1, \ldots m\}^{m}$ that encodes the permutation matrix $\mtx{P}$ by $\mathtt{P}[j] = i \Leftrightarrow \mtx{P_{ij}} = 1$.
Your function should not allocate any memory. 


\subsection*{(b) [5 pts]} 
Go to section (b) of the file \texttt{HW3\_your\_code.jl} and implement the unpivoted LU factorization. 
Check your code by ensuring that the assertions in section a + b of \texttt{HW3\_driver.jl} do not produce any errors.

\subsection*{(c) [5 pts]}
Generate families of random $m \times m$ matrices and vectors of length $m$. 
Plot as a function of the size $m$, the relative error of the solution obtained from your code in parts (a,b) and the growth factor introduced in problem 2.
Report the floating point type used by your program.
You can use the code provided in the second homework as a starting point for creating and saving plots.

\subsection*{(d) [5 pts]}
Go to section (d) of the file \texttt{HW3\_your\_code.jl} and implement the unpivoted LU factorization. 
Your code should pass the assertions in section (c) of \texttt{HW3\_driver.jl}.
Your function should take the matrix $A$ as an input to modify in place, and return an integer array $P$ according to the specifications of (a).
Repeat the experiment of (c) using the pivoted LU factorization. 


\subsection*{(e) [5 pts]}
Go to section (e) of the file \texttt{HW3\_driver.jl} and implement a function that takes an integer $m$ as an input and returns an $m \times m$ matrix as introduced in problem 2 (d).
Plot the error of the solution when solving equations in this matrix as a function of $m$. 
Compare the error to the built-in solution (the $\backslash$ operator). 
Draw your conclusions from this comparison. 

\bibliography{refs}

\end{document}

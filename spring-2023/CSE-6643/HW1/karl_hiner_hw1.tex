\documentclass[twoside,10pt]{article}
%=================================================
% Basics
%=================================================
\usepackage{fixltx2e} % Makes \( \) equation style robust, among other
                      % things. Must be the first package.


% Makes ligatured fonts searchable and copyable in pdf readers
\usepackage{cmap} % Load before fontenc 

% Always include these font encodings in your document 
% unless you have a very good reason.
\usepackage[T1]{fontenc}
\usepackage[utf8]{inputenc}

\usepackage{verbatim}

%=============
% Fonts
%=============

\usepackage{lmodern} % Improved version of computer modern
\usepackage[scale=0.88]{tgheros} % Helvetica clone for sans serif font


\newcommand\hmmax{2} % Default is 3.
\newcommand\bmmax{2} % Default is 4.

\usepackage{bm} % boldmath must be called after the package
\providecommand{\mathbold}[1]{\bm{#1}}

%=============
% AMS Packages and fonts
%=============
\usepackage{amsmath,amsbsy,amsgen,amscd,amsthm,amsfonts,amssymb} 

%=============
% Margins and paper size
%=============
\usepackage[centering,top=1.5in,bottom=1.2in,left=1.4in,right=1.4in]{geometry}
\usepackage{parskip}


%=============
% Section headings
%=============
\usepackage[sf,bf,compact]{titlesec}

%=============
% Tables and lists
%=============
\usepackage{booktabs,longtable,tabu} % Nice tables
\setlength{\tabulinesep}{1mm}
\usepackage[font=small,margin=10pt,labelfont={sf,bf},labelsep={space}]{caption}

%=============
% Code output
%=============
% \usepackage{listings}
% \usepackage{minted}




\usepackage{enumitem}
\setitemize{itemsep=0pt} 
\setenumerate{itemsep=0pt}
\setlist{labelindent=\parindent,%  % Recommended by enumitem package
  font=\sffamily}


%=============
% Hyperlink colors
%=============
\usepackage[usenames,dvipsnames]{xcolor}
\definecolor{steelblue}{HTML}{A1BDC7}
\definecolor{orange}{HTML}{D98C21}
\definecolor{silver}{HTML}{B0ABA8}
\definecolor{rust}{HTML}{B8420F}
\definecolor{seagreen}{HTML}{2E6B69}
\definecolor{joshua}{HTML}{FBDC7F}
\definecolor{darksky}{HTML}{154c79}

\colorlet{steelblue}{silver!30!white}
\colorlet{darkorange}{orange!85!black}
\colorlet{darksilver}{silver!85!black}
\colorlet{darksteelblue}{steelblue!85!black}
\colorlet{darkrust}{rust!85!black}
\colorlet{darkseagreen}{seagreen!85!black}

\usepackage{url}
\usepackage[colorlinks=true]{hyperref}
\hypersetup{linkcolor=darkrust}    
\hypersetup{citecolor=darkseagreen}      
\hypersetup{urlcolor=darksilver}     

%=============
% Microtype
%=============
\usepackage[final]{microtype} 

%=====================
% Header
%=====================
% \usepackage{fancyhdr}
% \usepackage{nopageno} % Gets rid of page number at the bottom
% \fancyhf{} % Clear header style
% \renewcommand{\headrulewidth}{0.5pt} % remove the header rule
% \pagestyle{fancy}
% \fancyhead[LE,RO]{\textsf{\small \thepage}}
% 
% \setlength{\headheight}{14pt}
%=====================
% Fix delimiters
%=====================

% Fixes \left and \right spacing issues. See discussion at
% http://tex.stackexchange.com/questions/2607/spacing-around-left-and-right
\let\originalleft\left
\let\originalright\right
\renewcommand{\left}{\mathopen{}\mathclose\bgroup\originalleft}
\renewcommand{\right}{\aftergroup\egroup\originalright}

%=================================================
% Math macros
%=================================================

%=============
% Generalities
%=============
\usepackage{mathtools}
\mathtoolsset{centercolon}  % Makes := typeset correctly for definitions

%%% Equation numbering
%\numberwithin{equation}{section} 

%%% Annotations
\newcommand{\notate}[1]{\textcolor{red}{\textbf{[#1]}}}

%==============
% Symbols
%==============
\let\oldphi\phi
\let\oldeps\epsilon

\renewcommand{\phi}{\varphi}
\renewcommand{\epsilon}{\varepsilon}
\newcommand{\eps}{\varepsilon}

%==============
% Constants
%==============

% Set constants upright
\newcommand{\cnst}[1]{\mathrm{#1}}  
\newcommand{\econst}{\mathrm{e}}

\newcommand{\zerovct}{\vct{0}} % Zero vector
\newcommand{\Id}{\mathbf{I}} % Identity matrix
\newcommand{\onemtx}{\bm{1}}
\newcommand{\zeromtx}{\bm{0}}

%==============
% Sets
%==============
\providecommand{\mathbbm}{\mathbb} % In case we don't load bbm

% Reals, complex, naturals
\newcommand{\R}{\mathbbm{R}}
\newcommand{\C}{\mathbbm{C}}
\newcommand{\K}{\mathbbm{K}}
\newcommand{\N}{\mathbbm{N}}

%==============
% Probability
%==============
\newcommand{\Prob}{\operatorname{\mathbbm{P}}}
\newcommand{\Expect}{\operatorname{\mathbb{E}}}

%==============
% Vectors and matrices 
%==============
\newcommand{\vct}[1]{\mathbold{#1}}
\newcommand{\mtx}[1]{\mathbold{#1}}

\newcommand{\mrange}{\operatorname{range}}
\newcommand{\mnull}{\operatorname{null}}



\newcommand{\eqn}[1]{\begin{equation}#1\end{equation}}
\newcommand{\aln}[1]{\begin{align}#1\end{align}}
\newcommand{\s}{\enspace}

\begin{document}

\title{CSE 6643 Homework 1}
\author{Karl Hiner}
\date{}
\maketitle

\section{Basics [25 pts]}

\subsection*{(a) [5 pts]}
Suppose that $\vct{v}_1, \vct{v}_2, \vct{v}_3, \vct{v}_4$ is a basis of the vector space $V \subset \R^n$. Prove that the list 
\begin{equation}
  \vct{v}_1 + \vct{v}_2, \vct{v}_2 + \vct{v}_3, \vct{v}_3 + \vct{v}_4, \vct{v}_4
\end{equation}
is also a basis of $V$.

We can prove these vectors form a basis of $V$ if we show that they are linearly independent and span the vector space.\\

\subsubsection*{(a1) Linear independence:}

To show that the given vectors are linearly independent, we first assume the vectors are linearly \textit{dependent}, and find the coefficients for the linear combination that equals the zero vector.
If the only solution is that the coefficients are all zero, then the vectors are linearly independent.

Suppose there exists $\{a_1, a_2, a_3, a_4\}, a_i \in \R, a_i \neq 0$ such that

\eqn{a_1(\vct{v}_1 + \vct{v}_2) + a_2(\vct{v}_2 + \vct{v}_3) + a_3(\vct{v}_3 + \vct{v}_4) + a_4\vct{v}_4 = \vct{0}.}

Rearranging terms,

\eqn{a_1\vct{v}_1 + (a_1 + a_2)\vct{v}_2 + (a_2 + a_3)\vct{v}_3 + (a_3 + a_4)\vct{v}_4 = \vct{0}.}

Since $\vct{v}_1, \vct{v}_2, \vct{v}_3, \vct{v}_4$ is a basis of $V$, we can rewrite this as

\eqn{a_1\vct{v}_1 + a_2\vct{v}_2 + a_3\vct{v}_3 + a_4\vct{v}_4 = \vct{0}.}

Since $\vct{v}_1, \vct{v}_2, \vct{v}_3, \vct{v}_4$ are linearly independent, we can conclude that $a_1 = a_2 = a_3 = a_4 = 0$.

Thus, the vectors $\vct{v}_1 + \vct{v}_2, \vct{v}_2 + \vct{v}_3, \vct{v}_3 + \vct{v}_4, \vct{v}_4$ are linearly independent.\\

\subsubsection*{(a2) Span:}

Since $(\vct{v}_1, \vct{v}_2, \vct{v}_3, \vct{v}_4)$ is a basis of $V$, any vector $\vct{u} \in V$ can be expressed as a linear combination with coefficients $\{a_1, a_2, a_3, a_4\}, a_i \in \R$.

If we can show that each vector in $L = (\vct{v}_1 + \vct{v}_2, \vct{v}_2 + \vct{v}_3, \vct{v}_3 + \vct{v}_4, \vct{v}_4)$ can be expressed in terms of $(\vct{v}_1, \vct{v}_2, \vct{v}_3, \vct{v}_4)$, then we will show that any vector $\vct{u} \in V$ can be also be expressed as a linear combination of the vectors in $L$:

\aln{
\vct{v}_1 + \vct{v}_2 &= (1)\vct{v}_1 + (1)\vct{v}_2\\
\vct{v}_2 + \vct{v}_3 &= (1)\vct{v}_2 + (1)\vct{v}_3\\
\vct{v}_3 + \vct{v}_4 &= (1)\vct{v}_3 + (1)\vct{v}_4\\
\vct{v}_4 &= (1)\vct{v}_4
}

Thus, $L$ spans $V$.

Therefore, the list $\vct{v}_1 + \vct{v}_2, \vct{v}_2 + \vct{v}_3, \vct{v}_3 + \vct{v}_4, \vct{v}_4$ is also a basis of V.

\subsection*{(b) [10 pts]}
For $U$ a subspace of the vector space $V \subset \R^n$ with $\dim(U) = \dim(V)$. Prove that $U = V$.

Since we are given that $U$ is a subspace of $V$, every vector in $U$ is also in $V$, so $U$ is a subset of $V$.

Since $\dim(U) = \dim(V)$, $U$ and $V$ have the same number of linearly independent vectors.
This, combined with the previously shown result that every vector in $U$ is also in $V$, implies that $U$ and $V$ have the same set of linearly independent vectors, and thus contain the same set of vectors.

\subsection*{(c) [10 pts]}
Show that the subspaces of $\R^3$ are precisely $\{0\}$, $\R^{3}$, all lines in $\R^3$ through the origin, and all planes in $\R^3$ through the origin.

First, we must show that these sets are indeed subspaces of $\R^3$.
For each set $\mathbb{U}$, we must show that the following conditions hold in order to show that it is a subspace of $\R^3$:

\begin{enumerate}
\item $\{0\} \in \mathbb{U}$
\item Closed under addition: $\vct{u}, \vct{v} \in \mathbb{U} \implies \vct{u} + \vct{v} \in \mathbb{U}$
\item Closed under scalar multiplication: $a \in \R, \vct{u} \in \mathbb{U} \implies a\vct{u} \in \mathbb{U}$
\end{enumerate}

Enumerating each of the four sets:
\begin{enumerate}
  \item $\mathbb{U} := \{0\}$: Since $\{0\} \in \{0\},$ and $dim(\{0\}) = 0$ (it is the only element in the set), it is trivially a subspace of $R^3$.
  \item $\mathbb{U} := $ All lines in $\R^3$ through the origin:
  \begin{itemize}
    \item Every line in $\R^3$ is defined by the set of points $\vct{u} = \vct{b} + c\vct{v}$, with $c \in \R, \vct{v} \neq \{0\}$.
    Since the set only includes lines passing through the origin, $\vct{u} = c\vct{v}$.
    \item $\{0\} \in \mathbb{U}$ by construction.
    \item For any $\vct{u} = a\vct{u'}$ and $\vct{v} = b\vct{v'}$, with $\vct{u'}, \vct{v'} \in R^3, \vct{u'}, \vct{v'} \neq \{0\}$ and $a,b \in \R$,
    \aln{
      c\vct{u} + d\vct{v} &= ca\vct{u'} + db\vct{v'}\\
      &= ca(\vct{u'} + \frac{ca}{db}\vct{v'})\\
      &= c'\vct{w} &(c' \equiv (ca) \in \R, \vct{w} \equiv (\vct{u'} + \frac{ca}{db}\vct{v'}) \in \R^3)
    }
    Thus, this set is closed under linear combinations (satisfying conditions 2 and 3). 
  \end{itemize}
  \item $\mathbb{U} := $ All planes in $\R^3$ through the origin: We could prove this algebraically using a similar approach to the case of lines through the origin.
    However, I will avoid repetition by noting that we can extend our proof over lines in $\R^3$ to a proof over planes in $\R^3$ by replacing our scalars ($c, d, a, b$) with 3-vectors (plane normals), without violating linearity.
    We could also make a geometric argument that multiplying a plane through the origin by a constant creates a new plane through the origin, as does adding together two planes through the origin.
  \item $\mathbb{U} := \R^3$: Since $\R^3$ is a space, and every space is its own subspace, $\R^3$ is a subspace of $\R^3$.
\end{enumerate}

Finally, we must show that any subspace in $\R^3$ must itself be a subspace of one of these four subspaces.

First, note that there can exist no subspace $V \subset \R^n$ with $\dim(V) > n$, since any $V$ with $\dim(V) > n$ must have more than $n$ basis vectors, and thus will necessarily have vectors that are not contained in $\R^n$.
Combining this with the result of $\textit{b}$ above, which holds that for any subspace $U$ of $V \subset \R^n$ with $\dim(U) = \dim(V), U = V$, we can conclude that there must be exactly $n + 1$ unique subspaces of $V \subset \R^n$ - one for each dimension $0, 1, ..., n$.

In this case, $V = \R^3$, so $n = 3$, and thus there are exactly $4$ subspaces of $\R^3$.
Since we have enumerated proofs for each of these four cases above, we are done.

\section{Norm Equivalencies [25 pts]}
In a finite-dimensional space, all norms are equivalent. In this problem, you will be asked to verify this theorem for some special norms. Prove the following inequalities. 

Let $\vct{x} \in \R^n$ be an $n$-dimensional vector. Let $\mtx{A} \in \R^{m \times n}$ be an $m \times n$ matrix. Then: 

\subsection*{(a) [10 pts]}
\begin{equation*}
  \|\vct{x}\|_{\infty} \leq \|\vct{x}\|_2 \leq \sqrt{n}\|\vct{x}\|_{\infty}.
\end{equation*}

\textit{Proof:}
\aln{
  \|\vct{x}\|_{\infty} &\leq &\|\vct{x}\|_2 &\leq &\sqrt{n}\|\vct{x}\|_{\infty}\\
  \max_{i}{|x_i|} &\leq &\sqrt{\sum_{i=1}^{n}{{x_i}^2}} &\leq &\sqrt{n}\max_{i}{|x_i|} \s\s &\text{(expanding definitions)}\\
  (\max_{i}{|x_i|})^2 &\leq &\sum_{i=1}^{n}{{x_i}^2} &\leq &n(\max_{i}{|x_i|})^2 \s\s &\text{(squaring all (positive) terms)}\\
  {x_m}^2 &\leq &\sum_{i=1}^{n}{{x_i}^2} &\leq &n{x_m}^2 \s\s &\text{(let $x_m \equiv \max_{i}{|x_i|}$)}
}

Consider the extreme case where $x_i \equiv x_m, \forall{x_i \in \vct{x}}$.
That is, all elements in $\vct{x}$ are the same element, and so its sum-of-squares is maximally large relative to its maximum absolute element, $x_m$.
Then,

\aln{
  {x_m}^2 &\leq &\sum_{i=1}^{n}{{x_m}^2} &\leq &n{x_m}^2 \s\s &\text{(subsituting $x_i = x_m$)}\\
  {x_m}^2 &\leq &n{x_m}^2 &\leq &n{x_m}^2 \s\s &\text{(QED for this case)}
}

Now, consider the other extreme, where $x_1 \not\equiv 0, x_i \equiv 0, \forall{i \neq 1}$.
That is, all elements in $\vct{x}$ are 0, except a single non-zero element, arbitrarily chosen to be at index $i=1$.
So, the absolute maximum element $x_m = |x_i|$, and the sum-of-squares of $\vct{x}$ is maximally small relative to $x_m$.
Then,

\aln{
  {x_m}^2 &\leq &\sum_{i=1}^{n}{{x_m}^2} &\leq &n{x_m}^2 \s\s\\
  {x_m}^2 &\leq &{x_m}^2 &\leq &n{x_m}^2 \s\s &\text{(QED for this case)}
}

All other cases will result in a $\sum_{i=1}^{n}{{x_i}^2}$ between these extremes.

QED.

\subsection*{(b) [7.5 pts]}
\begin{equation*}
  \|\mtx{A}\|_{\infty} \leq \sqrt{n}\|\mtx{A}\|_2.
\end{equation*}

\subsection*{(c) [7.5 pts]}
\begin{equation*}
  \|\mtx{A}\|_{2} \leq \sqrt{m}\|\mtx{A}\|_{\infty}.
\end{equation*}

\section{Perturbing [25 pts]}
For $\vct{u}, \vct{v} \in \K^m$, the matrix $\mtx{A} \coloneqq \Id + \vct{u}\vct{v}^{*}$ is called a \emph{rank-one} perturbation of the identity. 

\subsection*{(a) [15 pts]}
Show that if $\mtx{A}$ is nonsingular, then its inverse has the form $\mtx{A}^{-1} = \Id + \alpha \vct{u}\vct{v}^{*}$ for some scalar $\alpha$, and give an expression for $\alpha$.

\subsection*{(b) [5 pts]}
For what $\vct{u}$ and $\vct{v}$ is $\mtx{A}$ nonsingular? 

\subsection*{(c) [5 pts]} 
If $\mtx{A}$ is singular, what is $\mnull(\mtx{A})$?

\end{document}

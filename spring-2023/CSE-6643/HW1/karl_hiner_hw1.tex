\documentclass[twoside,10pt]{article}
%=================================================
% Basics
%=================================================
\usepackage{fixltx2e} % Makes \( \) equation style robust, among other
                      % things. Must be the first package.


% Makes ligatured fonts searchable and copyable in pdf readers
\usepackage{cmap} % Load before fontenc 

% Always include these font encodings in your document 
% unless you have a very good reason.
\usepackage[T1]{fontenc}
\usepackage[utf8]{inputenc}

\usepackage{verbatim}

%=============
% Fonts
%=============

\usepackage{lmodern} % Improved version of computer modern
\usepackage[scale=0.88]{tgheros} % Helvetica clone for sans serif font


\newcommand\hmmax{2} % Default is 3.
\newcommand\bmmax{2} % Default is 4.

\usepackage{bm} % boldmath must be called after the package
\providecommand{\mathbold}[1]{\bm{#1}}

%=============
% AMS Packages and fonts
%=============
\usepackage{amsmath,amsbsy,amsgen,amscd,amsthm,amsfonts,amssymb} 

%=============
% Margins and paper size
%=============
\usepackage[centering,top=1.5in,bottom=1.2in,left=1.4in,right=1.4in]{geometry}
\usepackage{parskip}


%=============
% Section headings
%=============
\usepackage[sf,bf,compact]{titlesec}

%=============
% Tables and lists
%=============
\usepackage{booktabs,longtable,tabu} % Nice tables
\setlength{\tabulinesep}{1mm}
\usepackage[font=small,margin=10pt,labelfont={sf,bf},labelsep={space}]{caption}

%=============
% Code output
%=============
% \usepackage{listings}
% \usepackage{minted}




\usepackage{enumitem}
\setitemize{itemsep=0pt} 
\setenumerate{itemsep=0pt}
\setlist{labelindent=\parindent,%  % Recommended by enumitem package
  font=\sffamily}


%=============
% Hyperlink colors
%=============
\usepackage[usenames,dvipsnames]{xcolor}
\definecolor{steelblue}{HTML}{A1BDC7}
\definecolor{orange}{HTML}{D98C21}
\definecolor{silver}{HTML}{B0ABA8}
\definecolor{rust}{HTML}{B8420F}
\definecolor{seagreen}{HTML}{2E6B69}
\definecolor{joshua}{HTML}{FBDC7F}
\definecolor{darksky}{HTML}{154c79}

\colorlet{steelblue}{silver!30!white}
\colorlet{darkorange}{orange!85!black}
\colorlet{darksilver}{silver!85!black}
\colorlet{darksteelblue}{steelblue!85!black}
\colorlet{darkrust}{rust!85!black}
\colorlet{darkseagreen}{seagreen!85!black}

\usepackage{url}
\usepackage[colorlinks=true]{hyperref}
\hypersetup{linkcolor=darkrust}    
\hypersetup{citecolor=darkseagreen}      
\hypersetup{urlcolor=darksilver}     

%=============
% Microtype
%=============
\usepackage[final]{microtype} 

%=====================
% Header
%=====================
% \usepackage{fancyhdr}
% \usepackage{nopageno} % Gets rid of page number at the bottom
% \fancyhf{} % Clear header style
% \renewcommand{\headrulewidth}{0.5pt} % remove the header rule
% \pagestyle{fancy}
% \fancyhead[LE,RO]{\textsf{\small \thepage}}
% 
% \setlength{\headheight}{14pt}
%=====================
% Fix delimiters
%=====================

% Fixes \left and \right spacing issues. See discussion at
% http://tex.stackexchange.com/questions/2607/spacing-around-left-and-right
\let\originalleft\left
\let\originalright\right
\renewcommand{\left}{\mathopen{}\mathclose\bgroup\originalleft}
\renewcommand{\right}{\aftergroup\egroup\originalright}

%=================================================
% Math macros
%=================================================

%=============
% Generalities
%=============
\usepackage{mathtools}
\mathtoolsset{centercolon}  % Makes := typeset correctly for definitions

%%% Equation numbering
%\numberwithin{equation}{section} 

%%% Annotations
\newcommand{\notate}[1]{\textcolor{red}{\textbf{[#1]}}}

%==============
% Symbols
%==============
\let\oldphi\phi
\let\oldeps\epsilon

\renewcommand{\phi}{\varphi}
\renewcommand{\epsilon}{\varepsilon}
\newcommand{\eps}{\varepsilon}

%==============
% Constants
%==============

% Set constants upright
\newcommand{\cnst}[1]{\mathrm{#1}}  
\newcommand{\econst}{\mathrm{e}}

\newcommand{\zerovct}{\vct{0}} % Zero vector
\newcommand{\Id}{\mathbf{I}} % Identity matrix
\newcommand{\onemtx}{\bm{1}}
\newcommand{\zeromtx}{\bm{0}}

%==============
% Sets
%==============
\providecommand{\mathbbm}{\mathbb} % In case we don't load bbm

% Reals, complex, naturals
\newcommand{\R}{\mathbbm{R}}
\newcommand{\C}{\mathbbm{C}}
\newcommand{\K}{\mathbbm{K}}
\newcommand{\N}{\mathbbm{N}}

%==============
% Probability
%==============
\newcommand{\Prob}{\operatorname{\mathbbm{P}}}
\newcommand{\Expect}{\operatorname{\mathbb{E}}}

%==============
% Vectors and matrices 
%==============
\newcommand{\vct}[1]{\mathbold{#1}}
\newcommand{\mtx}[1]{\mathbold{#1}}

\newcommand{\mrange}{\operatorname{range}}
\newcommand{\mnull}{\operatorname{null}}



\newcommand{\eqn}[1]{\begin{equation*}#1\end{equation*}}
\newcommand{\aln}[1]{\begin{align}#1\end{align}}
\newcommand{\s}{\enspace}

\begin{document}

\title{CSE 6643 Homework 1}
\author{Karl Hiner}
\date{}
\maketitle

\section{Basics [25 pts]}

\subsection*{(a) [5 pts]}
Suppose that $\vct{v}_1, \vct{v}_2, \vct{v}_3, \vct{v}_4$ is a basis of the vector space $V \subset \R^n$. Prove that the list 
\begin{equation}
  \vct{v}_1 + \vct{v}_2, \vct{v}_2 + \vct{v}_3, \vct{v}_3 + \vct{v}_4, \vct{v}_4
\end{equation}
is also a basis of $V$.

1. Since $\vct{v}_1, \vct{v}_2, \vct{v}_3, \vct{v}_4$ is a basis of the vector space $V \subset \R^n$, for all $\vct{u} \in V$,
\eqn{\vct{u} = a_1\vct{v}_1 + a_2\vct{v}_2 + a_3\vct{v}_3 + a_4\vct{v}_4, \s a_i \in \R.}
2. Rearranging terms:
\aln{
  \vct{u} &= a_1\vct{v}_1 + a_2\vct{v}_2 + a_3\vct{v}_3 + a_4\vct{v}_4\\
  &= a_1\vct{v}_1 + 2a_2\vct{v}_2 - a_2\vct{v}_2 + 2a_3\vct{v}_3 - a_3\vct{v}_3 + 2a_4\vct{v}_4 - a_4\vct{v}_4\\
  &= a_1(\vct{v}_1 + \frac{2a_2}{a_1}\vct{v}_2) + (-a_2)(\vct{v}_2 + \vct{v}_3) + ()(\vct{v}_3 + \vct{v}_4) + (-a_4)\vct{v}_4\\
}

\subsection*{(b) [10 pts]}
For $U$ a subspace of the vector space $V \subset \R^n$ with $\dim(U) = \dim(V)$. Prove that $U = V$.

\subsection*{(c) [10 pts]}
Show that the subspaces of $\R^3$ are precisely $\{0\}$, $\R^{3}$, all lines in $\R^3$ through the origin, and all planes in $\R^3$ through the origin.

\section{Norm Equivalencies [25 pts]}
In a finite-dimensional space, all norms are equivalent. In this problem, you will be asked to verify this theorem for some special norms. Prove the following inequalities. 

Let $\vct{x} \in \R^n$ be an $n$-dimensional vector. Let $\mtx{A} \in \R^{m \times n}$ be an $m \times n$ matrix. Then: 

\subsection*{(a) [10 pts]}
\begin{equation*}
  \|\vct{x}\|_{\infty} \leq \|x\|_2 \leq \sqrt{n}\|\vct{x}\|_{\infty}.
\end{equation*}

\subsection*{(b) [7.5 pts]}
\begin{equation*}
  \|\mtx{A}\|_{\infty} \leq \sqrt{n}\|\mtx{A}\|_2.
\end{equation*}

\subsection*{(c) [7.5 pts]}
\begin{equation*}
  \|\mtx{A}\|_{2} \leq \sqrt{m}\|\mtx{A}\|_{\infty}.
\end{equation*}

\section{Perturbing [25 pts]}
For $\vct{u}, \vct{v} \in \K^m$, the matrix $\mtx{A} \coloneqq \Id + \vct{u}\vct{v}^{*}$ is called a \emph{rank-one} perturbation of the identity. 

\subsection*{(a) [15 pts]}
Show that if $\mtx{A}$ is nonsingular, then its inverse has the form $\mtx{A}^{-1} = \Id + \alpha \vct{u}\vct{v}^{*}$ for some scalar $\alpha$, and give an expression for $\alpha$.

\subsection*{(b) [5 pts]}
For what $\vct{u}$ and $\vct{v}$ is $\mtx{A}$ nonsingular? 

\subsection*{(c) [5 pts]} 
If $\mtx{A}$ is singular, what is $\mnull(\mtx{A})$?

\end{document}

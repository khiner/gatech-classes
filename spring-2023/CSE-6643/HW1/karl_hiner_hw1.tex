\documentclass[twoside,10pt]{article}
\input{macro.tex}

\newcommand{\eqn}[1]{\begin{equation*}#1\end{equation*}}
\newcommand{\aln}[1]{\begin{align}#1\end{align}}
\newcommand{\s}{\enspace}

\begin{document}

\title{CSE 6643 Homework 1}
\author{Karl Hiner}
\date{}
\maketitle

\section{Basics [25 pts]}

\subsection*{(a) [5 pts]}
Suppose that $\vct{v}_1, \vct{v}_2, \vct{v}_3, \vct{v}_4$ is a basis of the vector space $V \subset \R^n$. Prove that the list 
\begin{equation}
  \vct{v}_1 + \vct{v}_2, \vct{v}_2 + \vct{v}_3, \vct{v}_3 + \vct{v}_4, \vct{v}_4
\end{equation}
is also a basis of $V$.

1. Since $\vct{v}_1, \vct{v}_2, \vct{v}_3, \vct{v}_4$ is a basis of the vector space $V \subset \R^n$, for all $\vct{u} \in V$,
\eqn{\vct{u} = a_1\vct{v}_1 + a_2\vct{v}_2 + a_3\vct{v}_3 + a_4\vct{v}_4, \s a_i \in \R.}
2. Rearranging terms:
\aln{
  \vct{u} &= a_1\vct{v}_1 + a_2\vct{v}_2 + a_3\vct{v}_3 + a_4\vct{v}_4\\
  &= a_1\vct{v}_1 + 2a_2\vct{v}_2 - a_2\vct{v}_2 + 2a_3\vct{v}_3 - a_3\vct{v}_3 + 2a_4\vct{v}_4 - a_4\vct{v}_4\\
  &= a_1(\vct{v}_1 + \frac{2a_2}{a_1}\vct{v}_2) + (-a_2)(\vct{v}_2 + \vct{v}_3) + ()(\vct{v}_3 + \vct{v}_4) + (-a_4)\vct{v}_4\\
}

\subsection*{(b) [10 pts]}
For $U$ a subspace of the vector space $V \subset \R^n$ with $\dim(U) = \dim(V)$. Prove that $U = V$.

\subsection*{(c) [10 pts]}
Show that the subspaces of $\R^3$ are precisely $\{0\}$, $\R^{3}$, all lines in $\R^3$ through the origin, and all planes in $\R^3$ through the origin.

\section{Norm Equivalencies [25 pts]}
In a finite-dimensional space, all norms are equivalent. In this problem, you will be asked to verify this theorem for some special norms. Prove the following inequalities. 

Let $\vct{x} \in \R^n$ be an $n$-dimensional vector. Let $\mtx{A} \in \R^{m \times n}$ be an $m \times n$ matrix. Then: 

\subsection*{(a) [10 pts]}
\begin{equation*}
  \|\vct{x}\|_{\infty} \leq \|x\|_2 \leq \sqrt{n}\|\vct{x}\|_{\infty}.
\end{equation*}

\subsection*{(b) [7.5 pts]}
\begin{equation*}
  \|\mtx{A}\|_{\infty} \leq \sqrt{n}\|\mtx{A}\|_2.
\end{equation*}

\subsection*{(c) [7.5 pts]}
\begin{equation*}
  \|\mtx{A}\|_{2} \leq \sqrt{m}\|\mtx{A}\|_{\infty}.
\end{equation*}

\section{Perturbing [25 pts]}
For $\vct{u}, \vct{v} \in \K^m$, the matrix $\mtx{A} \coloneqq \Id + \vct{u}\vct{v}^{*}$ is called a \emph{rank-one} perturbation of the identity. 

\subsection*{(a) [15 pts]}
Show that if $\mtx{A}$ is nonsingular, then its inverse has the form $\mtx{A}^{-1} = \Id + \alpha \vct{u}\vct{v}^{*}$ for some scalar $\alpha$, and give an expression for $\alpha$.

\subsection*{(b) [5 pts]}
For what $\vct{u}$ and $\vct{v}$ is $\mtx{A}$ nonsingular? 

\subsection*{(c) [5 pts]} 
If $\mtx{A}$ is singular, what is $\mnull(\mtx{A})$?

\end{document}

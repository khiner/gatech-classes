\documentclass{article}
%=================================================
% Basics
%=================================================
\usepackage{fixltx2e} % Makes \( \) equation style robust, among other
                      % things. Must be the first package.


% Makes ligatured fonts searchable and copyable in pdf readers
\usepackage{cmap} % Load before fontenc 

% Always include these font encodings in your document 
% unless you have a very good reason.
\usepackage[T1]{fontenc}
\usepackage[utf8]{inputenc}

\usepackage{verbatim}

%=============
% Fonts
%=============

\usepackage{lmodern} % Improved version of computer modern
\usepackage[scale=0.88]{tgheros} % Helvetica clone for sans serif font


\newcommand\hmmax{2} % Default is 3.
\newcommand\bmmax{2} % Default is 4.

\usepackage{bm} % boldmath must be called after the package
\providecommand{\mathbold}[1]{\bm{#1}}

%=============
% AMS Packages and fonts
%=============
\usepackage{amsmath,amsbsy,amsgen,amscd,amsthm,amsfonts,amssymb} 

%=============
% Margins and paper size
%=============
\usepackage[centering,top=1.5in,bottom=1.2in,left=1.4in,right=1.4in]{geometry}
\usepackage{parskip}


%=============
% Section headings
%=============
\usepackage[sf,bf,compact]{titlesec}

%=============
% Tables and lists
%=============
\usepackage{booktabs,longtable,tabu} % Nice tables
\setlength{\tabulinesep}{1mm}
\usepackage[font=small,margin=10pt,labelfont={sf,bf},labelsep={space}]{caption}

%=============
% Code output
%=============
% \usepackage{listings}
% \usepackage{minted}




\usepackage{enumitem}
\setitemize{itemsep=0pt} 
\setenumerate{itemsep=0pt}
\setlist{labelindent=\parindent,%  % Recommended by enumitem package
  font=\sffamily}


%=============
% Hyperlink colors
%=============
\usepackage[usenames,dvipsnames]{xcolor}
\definecolor{steelblue}{HTML}{A1BDC7}
\definecolor{orange}{HTML}{D98C21}
\definecolor{silver}{HTML}{B0ABA8}
\definecolor{rust}{HTML}{B8420F}
\definecolor{seagreen}{HTML}{2E6B69}
\definecolor{joshua}{HTML}{FBDC7F}
\definecolor{darksky}{HTML}{154c79}

\colorlet{steelblue}{silver!30!white}
\colorlet{darkorange}{orange!85!black}
\colorlet{darksilver}{silver!85!black}
\colorlet{darksteelblue}{steelblue!85!black}
\colorlet{darkrust}{rust!85!black}
\colorlet{darkseagreen}{seagreen!85!black}

\usepackage{url}
\usepackage[colorlinks=true]{hyperref}
\hypersetup{linkcolor=darkrust}    
\hypersetup{citecolor=darkseagreen}      
\hypersetup{urlcolor=darksilver}     

%=============
% Microtype
%=============
\usepackage[final]{microtype} 

%=====================
% Header
%=====================
% \usepackage{fancyhdr}
% \usepackage{nopageno} % Gets rid of page number at the bottom
% \fancyhf{} % Clear header style
% \renewcommand{\headrulewidth}{0.5pt} % remove the header rule
% \pagestyle{fancy}
% \fancyhead[LE,RO]{\textsf{\small \thepage}}
% 
% \setlength{\headheight}{14pt}
%=====================
% Fix delimiters
%=====================

% Fixes \left and \right spacing issues. See discussion at
% http://tex.stackexchange.com/questions/2607/spacing-around-left-and-right
\let\originalleft\left
\let\originalright\right
\renewcommand{\left}{\mathopen{}\mathclose\bgroup\originalleft}
\renewcommand{\right}{\aftergroup\egroup\originalright}

%=================================================
% Math macros
%=================================================

%=============
% Generalities
%=============
\usepackage{mathtools}
\mathtoolsset{centercolon}  % Makes := typeset correctly for definitions

%%% Equation numbering
%\numberwithin{equation}{section} 

%%% Annotations
\newcommand{\notate}[1]{\textcolor{red}{\textbf{[#1]}}}

%==============
% Symbols
%==============
\let\oldphi\phi
\let\oldeps\epsilon

\renewcommand{\phi}{\varphi}
\renewcommand{\epsilon}{\varepsilon}
\newcommand{\eps}{\varepsilon}

%==============
% Constants
%==============

% Set constants upright
\newcommand{\cnst}[1]{\mathrm{#1}}  
\newcommand{\econst}{\mathrm{e}}

\newcommand{\zerovct}{\vct{0}} % Zero vector
\newcommand{\Id}{\mathbf{I}} % Identity matrix
\newcommand{\onemtx}{\bm{1}}
\newcommand{\zeromtx}{\bm{0}}

%==============
% Sets
%==============
\providecommand{\mathbbm}{\mathbb} % In case we don't load bbm

% Reals, complex, naturals
\newcommand{\R}{\mathbbm{R}}
\newcommand{\C}{\mathbbm{C}}
\newcommand{\K}{\mathbbm{K}}
\newcommand{\N}{\mathbbm{N}}

%==============
% Probability
%==============
\newcommand{\Prob}{\operatorname{\mathbbm{P}}}
\newcommand{\Expect}{\operatorname{\mathbb{E}}}

%==============
% Vectors and matrices 
%==============
\newcommand{\vct}[1]{\mathbold{#1}}
\newcommand{\mtx}[1]{\mathbold{#1}}

\newcommand{\mrange}{\operatorname{range}}
\newcommand{\mnull}{\operatorname{null}}



\newcommand{\eqn}[1]{\begin{equation*}#1\end{equation*}}
\newcommand{\aln}[1]{\begin{align*}#1\end{align*}}
\newcommand{\?}{\stackrel{?}{=}}

\begin{document}

\title{CSE 6220 Homework 2}
\author{Karl Hiner, @khiner6}
\date{}
\maketitle

\section{}

Determine if the parallel prefix algorithm can be used to compute prefix sums of a sequence of $n$ numbers based on the binary operation $\bigoplus$ defined as:

\quad The parallel prefix algorithm can be used to compute the prefix sums of a sequence of $n$ numbers if and only if $\oplus$ is a binary associative operator.
\begin{enumerate}[label=(\alph*)]
  \item $a \oplus b = 2a + b$

  \quad Checking associativity,

  \aln{
    (a \oplus b) \oplus c &\? a \oplus (b \oplus c)\\
    2(2a + b) + c &\? 2a + (2b + c)\\
    4a + 2b + c &\neq 2a + 2b + c\\
  }
  This shows the binary operation defined as $a \oplus b = 2a + b$ \textit{is not} associative, and thus \textit{cannot} be used by the parallel prefix algorithm to compute prefix sums.

  \item $a \oplus b = \sqrt{a^2 + b^2}$

  \aln{
    (a \oplus b) \oplus c &\? a \oplus (b \oplus c)\\
    \sqrt{\left(\sqrt{a^2 + b^2}\right)^2 + c^2} &\? \sqrt{a^2 + \left(\sqrt{b^2 + c^2}\right)^2}\\
    \left(\sqrt{a^2 + b^2}\right)^2 + c^2 &\? a^2 + \left(\sqrt{b^2 + c^2}\right)^2\\
    a^2 + b^2 + c^2 &= a^2 + b^2 + c^2\\
  }
  This shows the binary operation defined as $a \oplus b = \sqrt{a^2 + b^2}$ \textit{is} associative, and thus \textit{can} be used by the parallel prefix algorithm to compute prefix sums.
\end{enumerate}

\section{}

In the game of Photosynthesis, points are given for trees that receive sunlight.
Consider $n$ trees $T_0,T_1,...,T_{n-1}$ planted along a single row of spaces, and sunlight is colinear with this row of trees.
The tree placement is modeled by an array $A$ of size $n$, where $A[i]$ denotes the height of the tree $T_i$.
A tree tall enough to get sunlight exposure scores photosynthesis points according to its height, i.e., $T_i$ is given $A[i]$ points.
However, a tree can also be blocked from sunlight by an earlier tree \textit{of equal height or taller}, in which case the blocked tree receives no points.

Design a parallel algorithm to compute the total number of points for a configuration given by $A$, and compute its runtime. 

\quad Assume the array $A$ is block-distributed across all nodes.
Let $p_k$ refer to the node holding the $n/p$ elements $\left[A[\frac{kn}{p}], ..., A[\frac{(k + 1)n}{p} - 1]\right] \equiv A_k$.
The algorithm can then be performed by the following steps:
\begin{enumerate}
  \item Compute the local maximum tree height, $\hat{A}_k = \max{\left(A_k\right)}$ for each node $p_k$. 
  This takes $O\left(\frac{n}{p}\right)$ time.
  \item Use a parallel prefix reduction with $\oplus = \max(\hat{A}_i, \hat{A}_j)$.
  (Note that the $\max$ operation is binary associative, and thus can be used for parallel prefix.)
  After the reduction, each node $p_k$ holds the maximum tree height, $\hat{A}_{<k}$, of all previous nodes $p_{j<k}$.
  Parallel prefix takes $O(\log{p})$ time, with communication cost $O(p)$.
  \item Compute the local total tree score for each node, as follows:
  \begin{itemize}
    \item Initialize the running score $s_k = 0$, and running height max $m_k = \hat{A}_{<k}$.
    \item For each tree height in the local list $A_k$, if $A_k[i] < m_k$, $s_k \mathrel{+}= A_k[i]$, else set $m_k = A_k[i]$.
  \end{itemize}
  \item Use the parallel sum algorithm to sum all local tree scores to find the total tree score, $S = \sum \limits_{k} s_k$.
  This final step takes $O(\log{p})$ time.
\end{enumerate}

TODO 1 \& 2 can actually be combined into one par prefix.
TODO add all times

\section{}

A sequence of nested parenthesis is said to be well-formed if 1) there are an equal number of left and right parenthesis, and 2) each right parenthesis is matched by a left parenthesis that occurs to its left in the sequence.
For example, ( ( ( ) ( ) ) ( ) ) is well-formed, but ( ) ) ( is not.

There is a nested parenthesis sequence of length $n$ distributed across $p$ processors using block decomposition.
Design a parallel algorithm to determine if it is well-formed and specify its run-time.

\section{}

Let $A$ be an array of $n$ elements and $L$ be a boolean array of the same size.
We want to assign a unique rank in the range $1, 2, ..., n$ to each element of $A$ such that for any $i < j$:
\begin{itemize}
\item If $L[i] = L[j]$, $A[i]$ has lower rank than $A[j]$.
\item If $L[i] = 0$ and $L[j] = 1$, $A[i]$ has lower rank than $A[j]$.
\item If $L[i] = 1$ and $L[j] = 0$, $A[j]$ has lower rank than $A[i]$.
\end{itemize}

Design a parallel algorithm to compute the ranks and specify its run-time.
(Hint: Think of $L$ as specifying labels.
Then, all elements with $0$ label receive lower ranks than any element with label $1$.
Within the same label, ranks are given in left to right order as per array $A$.)

\section{}

\textsl{Invent Segmented Parallel Prefix:}
Segmented parallel prefix is a generalization of the parallel prefix problem where the prefix sums need to be restarted at specified positions.
Consider array $X$ containing $n$ numbers and a boolean array $B$ of the same size.
We wish to compute prefix sums on $X$ but the sum resets at every position $i$ where $B[i] = 1$.
Formally, we wish to compute array $S$ of size $n$ such that

\aln{
  S[0] &= X[0]\\
  S[i] &=
    \begin{cases*}
      s[i-1] + X[i],&\text{if} B[i] = 0\\
      X[i],&\text{if} B[i] = 1
    \end{cases*}
}

Design parallel segmented prefix algorithm and specify its run-time.

(Hint: The problem can be transformed into a standard prefix sums problem.)

NOTE from Feb. 8 class: Can be solved using matrix raised to power of iteration - see class slides.

\end{document}

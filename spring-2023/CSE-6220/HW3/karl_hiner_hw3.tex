\documentclass{article}
\input{macro.tex}

\newcommand{\eqn}[1]{\begin{equation*}#1\end{equation*}}
\newcommand{\aln}[1]{\begin{align*}#1\end{align*}}

\begin{document}

\title{CSE 6220 Homework 3}
\author{Karl Hiner, @khiner6}
\date{}
\maketitle

\section{}
Draw a 8-element bitonic sorting circuit to sort elements in non-increasing order.
Use horizontal lines for the numbers and vertical lines to denote comparison operations.
Label the comparison operations as $\uparrow$ or $\downarrow$ where the direction of the arrowhead indicates
the destination of the smaller element.
Illustrate how the input 15, 2, 7, 4, 17, 3, 9, 1 is sorted using the diagram.

\section{}
The \textit{Bitonic Split} operation defined in class assumes that the bitonic sequence has
an even length.
Extend this operation to bitonic sequences of odd length.
Now, show how the bitonic sequence 5, 3, 4, 7, 10, 14, 17, 13, 8 can be converted into a sorted sequence using
repeated bitonic split operations.

\section{}
Give an algorithm to merge two sorted sequences of lengths $m$ and $n$, respectively.
You may assume that the input is an array of length $m+n$ with one sequence followed by the other, distributed across processors such that each processor has a subarray of size $\frac{m+n}{p}$.
What are the computation and communication times for your algorithm?
(You may assume the use of any permutation-style communication, not necessarily hypercubic.)

\section{}
Two algorithms are designed for the All-to-all communication primitive, and have
the following runtimes:

\aln{
  \text{Algorithm 1} &:&\Theta\left(\tau\log{p} + \mu m p \log{p}\right)\\
  \text{Algorithm 2} &:&\Theta\left(\tau p + \mu m p\right)
}
Determine which algorithm runs faster asymptotically as a function of the message size.

\section{}
Consider a tree of $n$ nodes and having a bounded degree (i.e., the number of children per node is bounded by a constant).
The tree is stored in an array $A$ of size $n$.
Each node also contains the indices at which its parent and children are stored in the array.
The array is distributed among processors using block decomposition, i.e., $P_i$ contains $A[i\frac{n}{p}],A[i\frac{n}{p} + 1, \dots, A[(i + 1)\frac{n}{p} - 1]].$
For each of the following operations, which communication primitive will you use and what is the runtime?

\begin{enumerate}[label=(\alph*)]
  \item Each node in the tree has $O(1)$ sized data that should be sent to its parent.
  \item Each node in the tree has $O(1)$ sized data that should be sent to all its children.
\end{enumerate}

\end{document}

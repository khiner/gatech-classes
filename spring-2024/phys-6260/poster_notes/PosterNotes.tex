\documentclass[12pt]{article}

\section{Background on Modal Vibration Models}

We briefly summarize the necessary background on modal vibration analysis here, and refer the reader to a suitable text \cite{shabana1990theory}. The linear elastodynamic equation for a finite element model \cite{zienkiewicz1977finite},

\begin{equation}
    M\ddot{u} + C\dot{u} + Ku = F, \tag{1}
\end{equation}

describes the displacements $u=u(t)$ of $N$ nodes within a volume. The displacement field $u$ is expanded in a modal displacement basis

\begin{equation}
    u(t) = \Phi q(t), \tag{2}
\end{equation}

where $\Phi$ denotes the model's modal matrix, a matrix whose $i$th column $\Phi_i$ represents the $i$th mode shape, and $q = q(t)$ are the corresponding modal amplitudes, i.e., $q_i$ is the modal amplitude of mode shape $\Phi_i$. An important property is that the modal matrix $\Phi$ is independent of time, and completely characterized by values at mesh vertices.

Substituting (2) into (1) and premultiplying by $\Phi^T$ yields

\begin{equation}
    M_q \ddot{q} + C_q \dot{q} + K_q q = Q, \tag{3}
\end{equation}

in which

\begin{equation}
    M_q = \Phi^T M \Phi = \operatorname{diag}(m_i), \tag{4}
\end{equation}

\begin{equation}
    K_q = \Phi^T K \Phi = \operatorname{diag}(k_i), \tag{5}
\end{equation}

\begin{equation}
    C_q = \Phi^T C \Phi, \tag{6}
\end{equation}

\begin{equation}
    Q = \Phi^T F. \tag{7}
\end{equation}

where all of $M_q$ and $K_q$ are diagonal matrices, but for general damping $C_q$ is dense. If we make the common assumption of proportional (Rayleigh) damping

\begin{equation}
    C = \alpha M + \beta K \Rightarrow C_q = \operatorname{diag}(\alpha m_i + \beta k_i),
\end{equation}

then the system of ODEs are completely decoupled by the modal transformation. This allows the motions due to individual modes to be computed independently and combined by linear superposition.

The system of decoupled ordinary differential equations may be written as

\begin{equation}
    \ddot{q}_i + 2\xi_i \omega_i \dot{q}_i + \omega_i^2 q_i = \frac{Q_i}{m_i}, \quad i = 1..n, \tag{8}
\end{equation}

where the undamped natural frequency of vibration is

\begin{equation}
    \omega_i = \sqrt{\frac{k_i}{m_i}} \quad \text{(in radians)}, \tag{9}
\end{equation}

and the dimensionless modal damping factor is

\begin{equation}
    \xi_i = \frac{c_i}{2m_i\omega_i} = \frac{1}{2} \left(\frac{\alpha}{\omega_i} + \beta\omega_i\right). \tag{10}
\end{equation}

We are interested in underdamped systems for which visible damped vibration occurs, and this corresponds to $\xi_i \in (0, 1)$.

Finally, for a system starting from rest at $t=0$, the solution for the $i$th mode due to forcing $Q_i(t)$ is

\begin{equation}
    q_i(t) = \int_{0}^{t} e^{-\xi_i \omega_i (t-\tau)} \sin \omega_{di} (t- \tau) \frac{Q_i(\tau)}{m_i \omega_{di}} \, d\tau, \tag{11}
\end{equation}

where the observed damped natural frequency is

\begin{equation}
    \omega_{di} = \omega_i \sqrt{1 - \xi_i^2}. \tag{12}
\end{equation}
